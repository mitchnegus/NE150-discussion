\documentclass{report}
% PACKAGES %
\usepackage[english]{} % Sets the language
\usepackage[margin=2cm]{geometry} % Sets the margin size
\usepackage{graphicx} % Enhanced package for including graphics/figures
\usepackage{float} % Allows figures and tables to be floats
\usepackage{amsmath} % Enhanced math package prepared by the American Mathematical Society
\usepackage{amssymb} % AMS symbols package
\usepackage{bm} % Allows you to use \bm{} to make any symbol bold
\usepackage{verbatim} % Allows you to include code snippets
\usepackage{setspace} % Allows you to change the spacing between lines at different points in the document
\usepackage{parskip} % Allows you alter the spacing between paragraphs
\usepackage{multicol} % Allows text division into multiple columns
\usepackage{units} % Allows fractions to be expressed diagonally instead of vertically
\usepackage{booktabs,multirow,multirow} % Gives extra table functionality
\usepackage{enumerate}
\newcommand{\tab}{\-\hspace{1.5cm}}

% Set path to figure image files
\graphicspath{ {fig/} }

\begin{document}

\begin{center}
\textbf{\large Nuclear Engineering 150 -- Discussion Section}\\ 
\textbf{Team Exercises \#12}
\end{center}

%%%%%%%%%%%%%%%%%%%%%%%%%%%%%%%%%% PROBLEM 1 %%%%%%%%%%%%%%%%%%%%%%%%%%%%%%%%%%
\section*{Problem 1}

Consider a sphere composed of a homogeneous multiplying medium, and surrounded by a reflecting shell of non-multiplying material with a thickness equal to half the radius of the sphere. Outside of the reflector is vacuum.
\begin{enumerate}[a.)]
\item Write the diffusion equation and boundary conditions describing this system.
\item Calculate the flux in both regions that are not vacuum.
\item Determine the reflector savings.
\end{enumerate}




\newpage
%%%%%%%%%%%%%%%%%%%%%%%%%%%%%%%%%% PROBLEM 2 %%%%%%%%%%%%%%%%%%%%%%%%%%%%%%%%%%
\section*{Problem 2}

Consider a bare sphere composed of a homogeneous multiplying medium.
\begin{enumerate}[a.)]
\item Give the steady-state, continuous energy diffusion equation. Assume that the diffusion coefficient ($D$) and the average neutrons produced from fission ($\nu$) are constant for all energies.
\item Derive the multigroup equation corresponding to the case where there are three energy groups. Assume that there is no upscattering, all groups are directly coupled (no inscattering), and fission is only induced by the slowest group while only producing neutrons in the fastest group.
\item Write the multigroup equation you found as a matrix-equation.
\end{enumerate}



\end{document}

