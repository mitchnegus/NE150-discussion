\documentclass{report}
% PACKAGES %
\usepackage[english]{} % Sets the language
\usepackage[margin=2cm]{geometry} % Sets the margin size
\usepackage{graphicx} % Enhanced package for including graphics/figures
\usepackage{float} % Allows figures and tables to be floats
\usepackage{amsmath} % Enhanced math package prepared by the American Mathematical Society
\usepackage{amssymb} % AMS symbols package
\usepackage{bm} % Allows you to use \bm{} to make any symbol bold
\usepackage{verbatim} % Allows you to include code snippets
\usepackage{setspace} % Allows you to change the spacing between lines at different points in the document
\usepackage{parskip} % Allows you alter the spacing between paragraphs
\usepackage{multicol} % Allows text division into multiple columns
\usepackage{units} % Allows fractions to be expressed diagonally instead of vertically
\usepackage{booktabs,multirow,multirow} % Gives extra table functionality
\usepackage{enumerate}
\newcommand{\tab}{\-\hspace{1.5cm}}

% Set path to figure image files
\graphicspath{ {fig/} }

\begin{document}

\begin{center}
\textbf{\large Nuclear Engineering 150 -- Discussion Section}\\ 
\textbf{Team Exercises \#10}
\end{center}

%%%%%%%%%%%%%%%%%%%%%%%%%%%%%%%%%% PROBLEM 1 %%%%%%%%%%%%%%%%%%%%%%%%%%%%%%%%%%
\section*{Problem 1}

Consider an infinite slab of material described by diffusion coefficient $D$ and macroscopic cross section $\Sigma_a$. The material extends infinitely in two dimensions, but has vacuum boundaries at $x=\pm a$. At $x=0$ is a uniformly distributed source plane with strength $s''$ [neutrons per area]. Find the flux in this geometry. (You may use the substitution $L = \sqrt{\frac{D}{\Sigma_a}}$.)

% See Lewis p. 143,148; http://www.nuceng.ca/ep4d3/text/3-diffusion-r1.pdf


\newpage
%%%%%%%%%%%%%%%%%%%%%%%%%%%%%%%%%% PROBLEM 2 %%%%%%%%%%%%%%%%%%%%%%%%%%%%%%%%%%
\section*{Problem 2}

Consider an infinite slab of material described by diffusion coefficient $D$ and macroscopic cross section $\Sigma_a$. The material extends infinitely in two dimensions, but has vacuum boundaries at $x=\pm a$. Inside the slab is a uniformly distributed source with strength $s'''$ [neutrons per volume]. Find the flux in this geometry. (You may use the substitution $L = \sqrt{\frac{D}{\Sigma_a}}$.)

% See Lewis p. 147



\end{document}

