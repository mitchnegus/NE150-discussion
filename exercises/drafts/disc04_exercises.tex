\documentclass{report}
% PACKAGES %
\usepackage[english]{} % Sets the language
\usepackage[margin=2cm]{geometry} % Sets the margin size
\usepackage{graphicx} % Enhanced package for including graphics/figures
\usepackage{float} % Allows figures and tables to be floats
\usepackage{amsmath} % Enhanced math package prepared by the American Mathematical Society
\usepackage{amssymb} % AMS symbols package
\usepackage{bm} % Allows you to use \bm{} to make any symbol bold
\usepackage{verbatim} % Allows you to include code snippets
\usepackage{setspace} % Allows you to change the spacing between lines at different points in the document
\usepackage{parskip} % Allows you alter the spacing between paragraphs
\usepackage{multicol} % Allows text division into multiple columns
\usepackage{units} % Allows fractions to be expressed diagonally instead of vertically
\usepackage{booktabs,multirow,multirow} % Gives extra table functionality
\usepackage{enumerate}
\newcommand{\tab}{\-\hspace{1.5cm}}

% Set path to figure image files
\graphicspath{ {fig/} }

\begin{document}

\begin{center}
\textbf{\large Nuclear Engineering 150 -- Discussion Section}\\ 
\textbf{Team Exercises \#4}
\end{center}

%%%%%%%%%%%%%%%%%%%%%%%%%%%%%%%%%% PROBLEM 1 %%%%%%%%%%%%%%%%%%%%%%%%%%%%%%%%%%
\section*{Problem 1}

A reactor contains 5\% (by atom) enriched uranium dioxide, which is 15\% of the entire core by volume.
\begin{enumerate}[a)]
\item Calculate the macroscopic cross section for this core if we were to treat it as a homogeneous volume. 
\item If the reactor were a cube with a side length of 4 m and a beam of $10^{15}$ thermal neutrons were incident on one face of the cube, how many neutrons would we expect to make it through to the other side?
\end{enumerate}

\begin{table}[htbp]
	\centering
	\begin{tabular}{|c|c|c|}
			\hline
			Nucleus		& Thermal $\sigma_{\text{t}}$ \\
			\hline
			$^{1}$H		&  20.8 b 		\\
			$^{16}$O	&  3.5 b 		\\
			$^{235}$U	&  607.5 b 		\\
			$^{238}$U	&  11.8 b 		\\
			\hline
			Compound	& $\rho$ 		\\
			\hline
			H$_2$O		& 1 g/cm$^3$	\\
			UO$_2$		& 10.4 g/cm$^3$	\\
			\hline
	\end{tabular}
	\label{tab:design-specs}
\end{table}



\newpage
%%%%%%%%%%%%%%%%%%%%%%%%%%%%%%%%%% PROBLEM 2 %%%%%%%%%%%%%%%%%%%%%%%%%%%%%%%%%%
\section*{Problem 2}

A neutron beam with an intensity of $2 \times 10^{12}\text{ neutrons/(cm}^2{\cdot}\text{s)}$ is incident on an unknown shielding material and has a beam spot of 5 cm$^3$. The shielding material has a thickness of 10 cm.
\begin{enumerate}[a)]
\item On average, $3.0\times10^9$ neutrons/s make it through the shield uncollided. What is the macroscopic cross section of the shield material?
\item What is the mean free path of a neutron in the shielding material?
\item If a single beam pulse is 10 $\mu$s, how many collisions are expected to take place in the shielding material?
\end{enumerate}



\end{document}

