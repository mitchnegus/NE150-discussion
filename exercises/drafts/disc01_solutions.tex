%% LaTeX template prepared by Sami Lewis for students in NE 101/210M, fall 2016
%% NOTES:
%%			- This is a simple, bare-bones template for the purposes of typing up homeworks. 
%%				There are many more advanced things you can add if you wish to do so.
%%			- You can make any changes to the template that you'd like.
%%			- Things in all caps are things you need to fill in. Make sure to change them
%%				all, including your name.
%%			-	There are lots of other, nicer free templates online too!
%%			- Run the template once without making any changes so that you can install
%%				all of the packages and see what the formatting looks like.

\documentclass{report}
% PACKAGES %
\usepackage[english]{} % Sets the language
\usepackage[margin=2cm]{geometry} % Sets the margin size
\usepackage{graphicx} % Enhanced package for including graphics/figures
\usepackage{float} % Allows figures and tables to be floats
\usepackage{amsmath} % Enhanced math package prepared by the American Mathematical Society
\usepackage{amssymb} % AMS symbols package
\usepackage{bm} % Allows you to use \bm{} to make any symbol bold
\usepackage{verbatim} % Allows you to include code snippets
\usepackage{setspace} % Allows you to change the spacing between lines at different points in the document
\usepackage{parskip} % Allows you alter the spacing between paragraphs
\usepackage{multicol} % Allows text division into multiple columns
\usepackage{units} % Allows fractions to be expressed diagonally instead of vertically
\usepackage{booktabs,multirow,multirow} % Gives extra table functionality
\usepackage{enumerate}
\newcommand{\tab}{\-\hspace{1.5cm}}

% Set path to figure image files
\graphicspath{ {fig/} }

\begin{document}

\begin{center}
\textbf{\large Nuclear Engineering 150 -- Discussion Section}\\ 
\textbf{Team Exercise Solutions \#1}

\-\\
{\small *Problems 1 \& 2 borrowed from Nuclear Engineering 101 homework problem sets, Fall 2016}
\end{center}


%%%%%%%%%%%%%%%%%%%%%%%%%%%%%%%%%% Problem 1 %%%%%%%%%%%%%%%%%%%%%%%%%%%%%%%%%%
\section*{Problem 1 Solution}



%%%%%%%%%%%%%%%%%%%%%%%%%%%%%%%%%% Problem 2 %%%%%%%%%%%%%%%%%%%%%%%%%%%%%%%%%%
\section*{Problem 2 Solution}
\begin{enumerate}[a)]
\item The $Q$-value of a reaction is given by:$$Q=[m(\text{x})+m(\text{X})-m(\text{y})-m(\text{Y})]c^{2}.$$

\begin{spacing}{1.75}	
$^{235}\text{\normalfont{U}}+n \rightarrow  \-\ ^{135}\text{\normalfont{I}}+ \! ^{98}\text{\normalfont{Y}}+3n$\\
	\tab $Q = [m(^{235}\text{\normalfont{U}})+m(n)-m(^{135}\text{\normalfont{I}})-m(^{98}\text{\normalfont{Y}})-3m(n)]c^{2}$\\
	\tab $Q = [235.043924\text{u}+1.008665\text{u}-134.910048\text{u}-97.922203\text{u}-3(1.008665\text{u})]c^{2}$\\
	\tab $Q = 0.194343\text{u} \cdot c^{2}$\\
	\tab $\boxed{Q = 181.031 \text{ MeV}}$ 
\end{spacing}


\item The binding energy $B(Z,A)$ for any nuclei can be found approximately from the equation:$$M(Z,A) = Zm(^{1}\text{H})+(A-Z)m_{n}-B(Z,A)/c^{2},$$ where $M(Z,A),\; Zm(^{1}\text{H})$ and $m_{n}$ are experimentally calculated values. Per nucleon, the binding energy can be expressed as: $$B_{A}(Z,A) = [Zm(^{1}\text{H})+(A-Z)m_{n}-M(Z,A)]c^{2}/A.$$

\begin{spacing}{1.75}

\textbf{$^{2}$H}\\
$B_{A}(1,2) = [m(^{1}\text{H})+(2-1)m_{n}-M(1,2)]c^{2}/2$\\
$B_{A}(1,2) = [(1.007825\text{u})+(1.008665\text{u})-(2.014102\text{u})]c^{2}/2$\\
$B_{A}(1,2) = 0.001194\text{u} \cdot c^{2}$\\
$\boxed{B_{A}(1,2) = 1.112\text{ MeV}}$\\
\textbf{$^{56}$Fe}\\
$B_{A}(26,56) = [26m(^{1}\text{H})+(56-26)m_{n}-M(26,56)]c^{2}/56$\\
$B_{A}(26,56) = [26(1.007825\text{u})+30(1.008665\text{u})-(55.934939\text{u})]c^{2}/56$\\
$B_{A}(26,56) = 0.009437\text{u} \cdot c^{2}$\\
$\boxed{B_{Z}(26,56) = 8.791\text{ MeV}}$\\
\textbf{$^{235}$U}\\
$B_{A}(92,235) = [92m(^{1}\text{H})+(235-92)m_{n}-M(92,235)]c^{2}/235$\\
$B_{A}(92,235) = [92(1.007825\text{u})+143(1.008665\text{u})-(235.043924\text{u})]c^{2}/235$\\
$B_{A}(92,235) = 0.00814924 u\text{u} \cdot c^{2}$\\
$\boxed{B_{Z}(92,235) = 7.591\text{ MeV}}$
\end{spacing}
\end{enumerate}



%%%%%%%%%%%%%%%%%%%%%%%%%%%%%%%%%% Problem 3 %%%%%%%%%%%%%%%%%%%%%%%%%%%%%%%%%%
\section*{Problem 3 Solution}
\begin{enumerate}[a)]
\item 
\begin{align}
\frac{dy}{dx}	&= -3y \\
\frac{dy}{y} 	& = -3\,dx \\
\int\frac{dy}{y}&= -3 \int \,dx \\
\ln y			&= -3 x + C \\
y				&= e^{-3x + C} \\
\end{align}
$$\boxed{y = Ce^{-3x}}$$

Solve the second order differential equation
$$ \frac{d^2 y}{dx^2} - A^2y = B $$
For the homogeneous equation, $\frac{d^2 y}{dx^2} - A^2y = 0$, try $y_c = X_1 e^{Ax} + X_2 e^{-Ax}$ as the complementary solution ($X_1$ and $X_2$ are constants). Then
$$ \frac{d^2 y}{dx^2} = X_1 A^2 e^{Ax} + X_2 A^2 e^{-Ax} $$
and
$$ (X_1 A^2 e^{Ax} + X_2 A^2 e^{-Ax}) - A^2(X_1 e^{Ax} + X_2 e^{-Ax}) = 0 $$
$$ A^2 (X_1 e^{Ax} + X_2 e^{-Ax}) - A^2 (X_1 e^{Ax} + X_2 e^{-Ax}) = 0 $$

This is \underline{true}.

For the inhomogeneous equation, $\frac{d^2 y}{dx^2} - A^2y = B$, $y_P = -B/A^2$ is the only particular solution satisfying the equation. The general solution is the sum of the complementary and particular solutions, $y = y_c + y_p$. 
$$ y = X_1 e^{Ax} + X_2 e^{-Ax} - \frac{B}{A^2} $$
Now we can solve for $X_1$ and $X_2$. 
$$ y(\frac{1}{A}) = X_1 e^{A(\frac{1}{A})} + X_2 e^{-A(\frac{1}{A})} - \frac{B}{A^2} = 0 $$
and
$$ y(-\frac{1}{A}) = X_1 e^{A(-\frac{1}{A})} + X_2 e^{A(\frac{1}{A})} - \frac{B}{A^2} = 0 $$
We can note that in these two equations, $X_1$ and $X_2$ can be interchanged freely, and so must be equal. We will say, $X_1 = X_2 = X$. Then, we have
$$ 0 = X e^{A(-\frac{1}{A})} + X e^{A(\frac{1}{A})} - \frac{B}{A^2} $$
$$ 0 = X (e^{-1} + e^{1}) - \frac{B}{A^2} $$
$$ X = \frac{B}{A^2(\frac{1}{e} + e)} $$
The we plug this $X$ into the final solution, which gives
$$ y = \frac{B}{A^2(\frac{1}{e} + e)} (e^{Ax} + e^{-Ax}) - \frac{B}{A^2} $$
\end{enumerate}
Note that $e^{Ax} + e^{-Ax}$ is similar in form to $\cosh(Ax)$ which we could have also used to solve this problem.



%%%%%%%%%%%%%%%%%%%%%%%%%%%%%%%%%% Problem 4 %%%%%%%%%%%%%%%%%%%%%%%%%%%%%%%%%%
\section*{Problem 4 Solution}


\tab\tab (4) $^{238}$U fission (high fast region) 					\\
\tab\tab (5) $^{239}$Pu fission (resonance peak)					\\
\tab\tab (1) $^{155}$Gd absorption (consistently high absorption)	\\
\tab\tab (2) $^{235}$U fission (high thermal fission cross section)	\\
\tab\tab (3) $^{238}$U absorption

\end{document}