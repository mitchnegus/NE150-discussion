\documentclass{report}
% PACKAGES %
\usepackage[english]{} % Sets the language
\usepackage[margin=2cm]{geometry} % Sets the margin size
\usepackage{graphicx} % Enhanced package for including graphics/figures
\usepackage{float} % Allows figures and tables to be floats
\usepackage{amsmath} % Enhanced math package prepared by the American Mathematical Society
\usepackage{amssymb} % AMS symbols package
\usepackage{breqn} % Allows equation breaking over multiple lines
\usepackage{bm} % Allows you to use \bm{} to make any symbol bold
\usepackage{verbatim} % Allows you to include code snippets
\usepackage{setspace} % Allows you to change the spacing between lines at different points in the document
\usepackage{parskip} % Allows you alter the spacing between paragraphs
\usepackage{multicol} % Allows text division into multiple columns
\usepackage{units} % Allows fractions to be expressed diagonally instead of vertically
\usepackage{booktabs,multirow,multirow} % Gives extra table functionality
\usepackage{enumerate} % Allows you to use customizable bullets
\usepackage{fancyhdr} % Allows customizable headers and footers
\usepackage{tikz} % Allows the creation of diagrams
	\usetikzlibrary{shapes.geometric, arrows}
	\tikzstyle{isotope} = [rectangle, 
					  minimum width=2cm, 
					  minimum height=1.25cm,
					  text centered, 
					  draw=black, 
					 ]
	\tikzstyle{decay} = [rectangle, 
					      rounded corners,
					      minimum width=2cm, 
					      minimum height=1.5cm,
					      text centered, 
					      draw=white, 
					     ]
	\tikzstyle{placeholder} = [rectangle,
					      minimum width=2cm,
					      minimum height=1cm,
					      draw=white,
					      ]
	\tikzstyle{arrow} = [thick,->,>=stealth]
	\tikzstyle{farrow} = [ultra thick,->,>=stealth]
	
% Set path to figure image files
\graphicspath{ {fig/} }

% Set some custom shortcuts
\newcommand{\tab}{\-\hspace{1.5cm}}
\newcommand{\lap}{\nabla^2}
\newcommand{\p}{\partial}

% Set the header on the first page and copyright on all pages
\fancypagestyle{FirstPage}{
\chead{\textbf{Nuclear Engineering 150 -- Discussion Section}}
\rfoot{\small \copyright~2019 Mitchell Negus}
}
\fancypagestyle{EveryPage}{
\rfoot{\small \copyright~2019 Mitchell Negus}
}
\pagestyle{EveryPage}


\begin{document}

\begin{center}
\textbf{\large Nuclear Engineering 150 -- Discussion Section}\\ 
\textbf{Team Exercise Solutions \#1}

\-\\
{\small *Problems 1 \& 2 borrowed from Nuclear Engineering 101 homework problem sets, Fall 2016}
\end{center}


%%%%%%%%%%%%%%%%%%%%%%%%%%%%%%%%%% PROBLEM 1 %%%%%%%%%%%%%%%%%%%%%%%%%%%%%%%%%%
\section*{Problem 1}
The radioactive isotope $^{233}$Pa can be produced following neutron capture by $^{232}$Th when the resulting $^{233}$Th decays to $^{233}$Pa. In the neutron flux of a typical reactor, neutron capture in 1 g of $^{232}$Th produces $^{233}$Th at of a rate of $2.0 \times 10^{11}\text{ s}^{-1}$.
\begin{enumerate}[a)]
\item What are the activities (in Ci) of $^{233}$Th and $^{233}$Pa after this sample is irradiated for 1.5 hours?
\item The sample is then placed in storage with no further irradiation so that the $^{233}$Th can decay away. What are
the activities (in Ci) of $^{233}$Th and $^{233}$Pa after 48 hours of storage?
\item The decay of $^{233}$Pa results in $^{233}$U, which is also radioactive. After the above sample has been stored for 1 year what is the $^{233}$U activity in Ci? (Hint: it should not be necessary to set up an additional differential equation to find the $^{233}$U activity.)
\end{enumerate}

\begin{table}[htbp]
	\centering
	\begin{tabular}{|c|c|}
			\hline
			Nucleus		&	Half-life \\
			\hline
			$^{233}$Th	&  $22.3$ min\\
			$^{233}$Pa	&  $27.0$ days\\
			$^{233}$U	&  $1.592 \times 10^5$ yr\\
			\hline
	\end{tabular}
	\label{tab:design-specs}
\end{table}
\begin{center}$1\text{ Ci} = 3.7 \times 10^{10}\text{ s}^{-1}$\end{center}



\section*{Problem 1 Solution}


For all parts of this problem, let $R$ be the rate of neutron capture by 1 g of $^{232}$Th, $2.0 \times 10^{11} \;\text{s}^{-1}$.\\

\begin{enumerate}[a)]

\item

First, convert all half-lives and irradiation times to seconds for consistency.\\ 
\-\\
\tab $\lambda_{\text{Th}} = \frac{\ln{2}}{1338\text{s}} = 5.18\times10^{-4}\text{s}^{-1}$; \tab $\lambda_{\text{Pa}}=\frac{\ln{2}}{2332800s}=2.971\times10^{-7}\text{s}^{-1}$; \tab 1.5 hr = 5400 s\\

\textbf{Thorium-233}\\
\-\\
The rate of change of the quantity of $^{233}$Th, $\frac{dN_{\text{th}}}{dt}$ is given by the production rate of $^{233}$Th, $R$, minus the decay rate (activity) of $^{233}$Th, $\lambda_{\text{Th}}N_{\text{Th}}$.
$$\frac{dN_{\text{Th}}}{dt} = R - \lambda_{\text{Th}}N_{\text{Th}}$$
We solve the differential equation, manipulating the equation so that the left side is only dependent on $N_{\text{Th}}$ and the right side is only dependent on $dt$. Then we integrate:
$$\int{\frac{dN_{\text{Th}}}{R-\lambda_{\text{Th}}N_{\text{Th}}}} = \int{dt}$$
$$\frac{-1}{\lambda_{\text{Th}}}[\ln(R-\lambda_{\text{Th}}N_{\text{Th}})] = t + C,\quad C=\text{const.}$$
$$R-\lambda_{\text{Th}}N_{\text{Th}} = e^{-\lambda_{\text{Th}}t -\lambda_{\text{Th}}C} = e^{-\lambda_{\text{Th}}t} e^{-\lambda_{\text{Th}}C}$$
We note that since $C$ is an arbitrary constant and $\lambda_{\text{Th}}$ is fixed, we could also say $e^{-\lambda_{\text{Th}}C}$ is an arbitrary constant, and just call it $C$ instead.
$$ R-\lambda_{\text{Th}}N_{\text{Th}} = Ce^{-\lambda_{\text{Th}}t} $$
We solve for $N_{\text{Th}}$ (explicitly including $N_{\text{Th}}$'s dependence on $t$), and get
$$ N_{\text{Th}}(t) = \frac{R - Ce^{-\lambda_{\text{Th}}t}}{\lambda_{\text{Th}}} .$$
At $t=0,\; N_{\text{Th}}(0) = \frac{R - C}{\lambda_{\text{Th}}}=0,$ since no $^{233}$\text{Th} has been formed. We find $C= R$, and use this in the general equation:
$$ N_{\text{Th}}(t) = R\frac{1 - e^{-\lambda_{\text{Th}}t}}{\lambda_{\text{Th}}}. $$
With this function of $N_{\text{Th}}$, we can determine the activity as a function of time, knowing that
$$ \mathcal{A}_{\text{Th}}(t) = \lambda_{\text{Th}}N_{\text{Th}}(t). $$
Substituting, we find
$$ \mathcal{A}_{\text{Th}}(t) = R(1-e^{-\lambda_{\text{Th}}t}) .$$
Using the numerical values for $R$, $\lambda_{\text{Th}}$, and $t$,
$$ \mathcal{A}_{\text{Th}}(1.5\text{ hr}) =(2.0\times10^{11}s^{-1})(1-e^{(-5.18\times10^{-4}\text{s}^{-1})(5400s)}) $$
$$ \mathcal{A}_{\text{Th}}(1.5\text{ hr}) = 1.878\times10^{11}\text{ Bq} .$$
Finally, we convert this to Curies,
$$\boxed{ \mathcal{A}_{\text{Th}}(1.5\text{ hr}) = 5.076\text{ Ci} }.$$

\textbf{Protactinium-233}\\
\-\\
We follow a similar procedure for $^{233}$Pa, noting that the production rate of $^{233}$Pa is just the activity of $^{233}$Th as it decays into $^{233}$Pa, $\mathcal{A}_{\text{Th}}$.
$$\frac{dN_{\text{Pa}}}{dt} = \mathcal{A}_{\text{Th}} - \lambda_{\text{Pa}}N_{\text{Pa}}$$
From above, we can substitute our function for $\mathcal{A}_{\text{Th}}(t)$,
$$\frac{dN_{\text{Pa}}}{dt} = R(1-e^{-\lambda_{\text{Th}}t}) - \lambda_{\text{Pa}}N_{\text{Pa}}$$
Since we cannot separate both sides to be dependent only on a single differential, we must try a different method of integration. We will use integrating factors. Still, we start in a similar fashion: collecting the terms dependent on $N_{\text{Pa}}$ on the same side.
$$ \frac{dN_{\text{Pa}}}{dt}+\lambda_{\text{Pa}}N_{\text{Pa}} = R(1-e^{-\lambda_{\text{Th}}t}) $$
The method of integrating factors suggests that we multiply both sides by an arbitrary exponential. We will use $e^{\lambda_{\text{Pa}}t}$.
$$ e^{\lambda_{\text{Pa}}t}\frac{dN_{\text{Pa}}}{dt} + \lambda_{\text{Pa}}e^{\lambda_{\text{Pa}}t}N_{\text{Pa}} = e^{\lambda_{\text{Pa}}t}R(1-e^{-\lambda_{\text{Th}}t}) $$
We can now observe that the left side of the equation appears to be the result of the product rule when the time derivative of $e^{\lambda_{\text{Pa}}t}N_{\text{Pa}}$ is found. We can then write the equation as
$$\frac{d}{dt}(e^{\lambda_{\text{Pa}}t}N_{\text{Pa}}) = e^{\lambda_{\text{Pa}}t}R(1-e^{-\lambda_{\text{Th}}t})$$
Moving the $dt$ term to the right side of the equation and using the distributive property, we have
$$ d(e^{\lambda_{\text{Pa}}t}N_{\text{Pa}}) = R(e^{\lambda_{\text{Pa}}t}-e^{\lambda_{\text{Pa}}t}e^{-\lambda_{\text{Th}}t})dt $$
or more simply (by exploiting properties of exponents)
$$ d(e^{\lambda_{\text{Pa}}t}N_{\text{Pa}}) = R(e^{\lambda_{\text{Pa}}t}-e^{\lambda_{\text{Pa}}t-\lambda_{\text{Th}}t})dt .$$
We integrate both sides,
$$ \int{d\left(e^{\lambda_{\text{Pa}}t}N_{\text{Pa}}\right)} = \int{ R(e^{\lambda_{\text{Pa}}t}-e^{\lambda_{\text{Pa}}t-\lambda_{\text{Th}}t})dt}, $$
separate the integral on the right side,
$$ \int{d\left(e^{\lambda_{\text{Pa}}t}N_{\text{Pa}}\right)} = R\int{ e^{\lambda_{\text{Pa}}t}dt}-R\int{e^{(\lambda_{\text{Pa}}-\lambda_{\text{Th}})t}dt} $$
and find
$$e^{\lambda_{\text{Pa}}t}N_{\text{Pa}} = \frac{R}{\lambda_{\text{Pa}}}e^{\lambda_{\text{Pa}}t}-\frac{R}{\lambda_{\text{Pa}}-\lambda_{\text{Th}}}e^{(\lambda_{\text{Pa}}-\lambda_{\text{Th}})t} +\;C,\; C=\text{const.}$$
Now we factor out the integrating factor back out from both sides and note explicitly the time dependence of $N_{\text{Pa}}$,
$$N_{\text{Pa}}(t) = \frac{R}{\lambda_{\text{Pa}}}-\frac{R}{\lambda_{\text{Pa}}-\lambda_{\text{Th}}}e^{-\lambda_{\text{Th}}t}+Ce^{-\lambda_{\text{Pa}}t}$$
At $t=0$,
$$ N_{\text{Pa}}(0) = \frac{R}{\lambda_{\text{Pa}}}-\frac{R}{\lambda_{\text{Pa}}-\lambda_{\text{Th}}}+C = 0, $$
 since no $^{233}$\text{Pa} has been formed. Solving for $C$, we find $C = \frac{R}{\lambda_{\text{Pa}}-\lambda_{\text{Th}}}-\frac{R}{\lambda_{\text{Pa}}}$. We plug this back into our equation above, and have the solution for $N_{\text{Pa}}(t)$:
$$N_{\text{Pa}}(t) = \frac{R}{\lambda_{\text{Pa}}}-\frac{R}{\lambda_{\text{Pa}}-\lambda_{\text{Th}}}e^{-\lambda_{\text{Th}}t}+\left( \frac{R}{\lambda_{\text{Pa}}-\lambda_{\text{Th}}}-\frac{R}{\lambda_{\text{Pa}}} \right)e^{-\lambda_{\text{Pa}}t}$$
and simplifying
$$ N_{\text{Pa}}(t) = \frac{R}{\lambda_{\text{Pa}}}(1-e^{\lambda_{\text{Pa}}t})+ \left(\frac{R}{\lambda_{\text{Pa}}-\lambda_{\text{Th}}}\right)(e^{-\lambda_{\text{Pa}}t}-e^{-\lambda_{\text{Th}}t}) .$$
With this function of $N_{\text{Pa}}$, we can determine the activity as a function of time, knowing that
$$ \mathcal{A}_{\text{Pa}}(t) = \lambda_{\text{Pa}}N_{\text{Pa}}(t) .$$
Substituting, we find
$$ \mathcal{A}_{\text{Pa}}(t) = R(1-e^{-\lambda_{\text{Pa}}t})+ \left(\frac{R \, \lambda_{\text{Pa}}}{\lambda_{\text{Pa}}-\lambda_{\text{Th}}}\right)(e^{-\lambda_{\text{Pa}}t}-e^{-\lambda_{\text{Th}}t}) .$$
Using the numerical values for $R$, $\lambda_{\text{Th}}$, $\lambda_{\text{Pa}}$, and $t$,
\begin{dmath*}
\mathcal{A}_{\text{Pa}}(1.5\text{ hr}) = (2.0\times10^{11}\text{ s}^{-1})\left(1-e^{(-2.971\times10^{-7}\text{s}^{-1})(5400\text{s})}\right)+ \left(\frac{2.0\times10^{11}\text{ s}^{-1}(2.791\times10^{-7}\text{s}^{-1})}{2.971\times10^{-7}\text{s}^{-1}-5.18\times10^{-4}\text{s}^{-1}}\right)\left(e^{(-2.971\times10^{-7}\text{s}^{-1})(5400\text{s})}-e^{(-5.18\times10^{-4}\text{s}^{-1})(5400\text{s})}\right)
\end{dmath*}
$$ \mathcal{A}_{\text{Pa}}(1.5\text{ hr}) = 2.195 \times 10^{8}\text{ Bq} $$
Finally, we convert this to Curies, 
$$ \boxed{\mathcal{A}_{\text{Pa}}(1.5\text{ hr}) = 0.006\text{ Ci}}. $$

\item

Let's say that the 1.5 hour mark is now given by $t=t_0=1.5\text{ hr}$. We also note that 48 hours = 172,800 seconds.

\textbf{Thorium-233}\\
\-\\
Without irradiation, the rate of change of the quantity of $^{233}$Th is now just the decay rate. 
$$ \frac{dN_{\text{Th}}}{dt} = -\lambda_{\text{Th}}N_{\text{Th}} $$
We separate the equation and integrate, arriving at the standard exponential decay formula, now including the explicit time dependence.
$$ \int_{N_{\text{Th}}(t_0)}^{N_{\text{Th}}(t)} \frac{-dN_{\text{Th}}'}{\lambda_{\text{Th}}N_{\text{Th}}'} = \int_{t_0}^{t} dt' $$
$$ \frac{-1}{\lambda_{\text{Th}}}\left[ \ln{N_{\text{Th}}} \right]_{N_{\text{Th}(t_0)}}^{N_{\text{Th}}(t)} = \left[t'\right]_{t_0}^{t} $$
$$ \ln\frac{N_{\text{Th}}(t)}{N_{\text{Th}}(t_0)}  = -\lambda_{\text{Th}}(t-t_0) $$
$$ \frac{N_{\text{Th}}(t)}{N_{\text{Th}}(t_0)}  = e^{-\lambda_{\text{Th}}(t-t_0)} $$
and we have 
$$ N_{\text{Th}}(t) = N_{\text{Th}}(t_0) e^{-\lambda_{\text{Th}}\left(t-t_0\right)} $$
Given the definition of activity as $\mathcal{A} = \lambda N$ and noting $\lambda_{\text{Th}} N_{\text{Th}}(t_0) = \mathcal{A}_{\text{Th}}(t_0)$, we can write the activity of $^{233}$Th as
\begin{align*}
\mathcal{A}_{\text{Th}}(t)	&= \lambda_{\text{Th}} N_{\text{Th}}(t) \\
							&= \lambda_{\text{Th}} N_{\text{Th}}(t_0) e^{-\lambda_{\text{Th}}(t-t_0)} \\ 
							&= \mathcal{A}_{\text{Th}}(t_0) e^{-\lambda_{\text{Th}}(t-t_0)}
\end{align*}
Using the numerical values for $\lambda_{\text{Th}}$, $t$, and our answer from part (a) for the activity at $t_0=1.5$ hr, we find
$$ \mathcal{A}_{\text{Th}}(49.5\text{ hr}) = (5.076 \text{ Ci})e^{-5.18\times10^{-4}\text{s}^{-1} (172800\text{s})} $$
$$ \boxed{\mathcal{A}_{\text{Th}}(49.5\text{ hr}) = 6.787\times10^{-39}\text{ Ci}} $$

{\small Note: we could also have assumed that since 48 hours is many (more than 100) times longer than the half-life of $^{233}$Th, that the activity would be approximately zero.}
\-\\
\-\\

\textbf{Protactinium-233}\\
\-\\
We follow the example in part (a) for $^{233}$Pa, again using the activity of $^{233}$Th as the production rate of $^{233}$Pa.
$$ \frac{dN_{\text{Pa}}}{dt} = \mathcal{A}_{\text{Th}} - \lambda_{\text{Pa}}N_{\text{Pa}}. $$
We collect terms dependent on $N_{\text{Pa}}$ on one side,
$$ \frac{dN_{\text{Pa}}}{dt} + \lambda_{\text{Pa}}N_{\text{Pa}} = \mathcal{A}_{\text{Th}}(t_0) e^{-\lambda_{\text{Th}}(t-t_0)}, $$
multiply both sides by arbitrary exponential $e^{\lambda_{\text{Pa}}t}$,
$$ e^{\lambda_{\text{Pa}}t}\frac{dN_{\text{Pa}}}{dt} + \lambda_{\text{Pa}}e^{\lambda_{\text{Pa}}t}N_{\text{Pa}} = e^{\lambda_{\text{Pa}}t}\mathcal{A}_{\text{Th}}(t_0) e^{-\lambda_{\text{Th}}(t-t_0)}, $$
note that the left side is the result of the product rule when $\frac{d}{dt}$ is taken on $e^{\lambda_{\text{Pa}}t}N_{\text{Pa}}$
$$ \frac{d}{dt} (e^{\lambda_{\text{Pa}}t}N_{\text{Pa}}) = e^{\lambda_{\text{Pa}}t}\mathcal{A}_{\text{Th}}(t_0) e^{-\lambda_{\text{Th}}(t-t_0)}, $$
multiply by the differential, $dt$,
$$ d(e^{\lambda_{\text{Pa}}t}N_{\text{Pa}}) = e^{\lambda_{\text{Pa}}t}\mathcal{A}_{\text{Th}}(t_0) e^{-\lambda_{\text{Th}}(t-t_0)} dt ,$$
and rearrange,
$$ d(e^{\lambda_{\text{Pa}}t}N_{\text{Pa}}) = \mathcal{A}_{\text{Th}}(t_0) e^{(\lambda_{\text{Pa}} - \lambda_{\text{Th}})t + \lambda_{\text{Th}}t_0} dt .$$
Then we integrate,
$$ \int_{e^{\lambda_{\text{Pa}}t_0}N_{\text{Pa}}(t_0)}^{e^{\lambda_{\text{Pa}}t}N_{\text{Pa}}(t)} d(e^{\lambda_{\text{Pa}}t}N_{\text{Pa}}') = \mathcal{A}_{\text{Th}}(t_0)e^{\lambda_{\text{Th}}t_0} \int_{t_0}^{t} e^{(\lambda_{\text{Pa}} - \lambda_{\text{Th}})t'} dt' ,$$
and find,
$$ \left[e^{\lambda_{\text{Pa}}t}N_{\text{Pa}}'\right]_{e^{\lambda_{\text{Pa}}t_0}N_{\text{Pa}}(t_0)}^{e^{\lambda_{\text{Pa}}t}N_{\text{Pa}}(t)} = \frac{\mathcal{A}_{\text{Th}}(t_0)e^{\lambda_{\text{Th}}t_0}}{\lambda_{\text{Pa}}-\lambda_{\text{Th}}} \left[ e^{(\lambda_{\text{Pa}}-\lambda_{\text{Th}})t'}\right]_{t_0}^{t} $$
$$ e^{\lambda_{\text{Pa}}t}N_{\text{Pa}}(t) - e^{\lambda_{\text{Pa}}t_0}N_{\text{Pa}}(t_0) = \frac{\mathcal{A}_{\text{Th}}(t_0)e^{\lambda_{\text{Th}}t_0}}{\lambda_{\text{Pa}}-\lambda_{\text{Th}}} \left[ e^{(\lambda_{\text{Pa}}-\lambda_{\text{Th}})t} - e^{(\lambda_{\text{Pa}}-\lambda_{\text{Th}})t_0} \right] $$
Factoring out the integrating factor, 
$$ N_{\text{Pa}}(t) - e^{-\lambda_{\text{Pa}}(t - t_0)}N_{\text{Pa}}(t_0) = \frac{\mathcal{A}_{\text{Th}}(t_0)e^{\lambda_{\text{Th}}t_0}}{\lambda_{\text{Pa}}-\lambda_{\text{Th}}} \left[ e^{-\lambda_{\text{Th}}t} - e^{-\lambda_{\text{Pa}}t}e^{(\lambda_{\text{Pa}}-\lambda_{\text{Th}})t_0} \right] $$
and solve for $ N_{\text{Pa}}(t)$,
$$ N_{\text{Pa}}(t) = \frac{\mathcal{A}_{\text{Th}}(t_0)e^{\lambda_{\text{Th}}t_0}}{\lambda_{\text{Pa}}-\lambda_{\text{Th}}} \left[ e^{-\lambda_{\text{Th}}t} - e^{-\lambda_{\text{Pa}}t}e^{(\lambda_{\text{Pa}}-\lambda_{\text{Th}})t_0} \right] + e^{-\lambda_{\text{Pa}}(t-t_0)}N_{\text{Pa}}(t_0) .$$
Rearranging,
$$ N_{\text{Pa}}(t) = \frac{\mathcal{A}_{\text{Th}}(t_0)}{\lambda_{\text{Pa}}-\lambda_{\text{Th}}} \left[ e^{-\lambda_{\text{Th}}(t - t_0)} - e^{-\lambda_{\text{Pa}}(t-t_0)} \right] + e^{-\lambda_{\text{Pa}}(t - t_0)}N_{\text{Pa}}(t_0) .$$
and so
$$ N_{\text{Pa}}(t) = \frac{\mathcal{A}_{\text{Th}}(t_0)}{\lambda_{\text{Pa}}-\lambda_{\text{Th}}} e^{-\lambda_{\text{Th}}(t-t_0)} + \left( N_{\text{Pa}}(t_0) - \frac{\mathcal{A}_{\text{Th}}(t_0)}{\lambda_{\text{Pa}}-\lambda_{\text{Th}}} \right) e^{-\lambda_{\text{Pa}}(t-t_0)} $$
Given the definition of activity as $\mathcal{A} = \lambda N$ and noting $\lambda_{\text{Pa}} N_{\text{Pa}}(t_0) = \mathcal{A}_{\text{Pa}}(t_0)$, we can write the activity of $^{233}$Pa as
\begin{align*}
\mathcal{A}_{\text{Pa}}(t)	&= \lambda_{\text{Pa}} N_{\text{Pa}}(t) \\
							&=\lambda_{\text{Pa}} \left(\frac{\mathcal{A}_{\text{Th}}(t_0)}{\lambda_{\text{Pa}}-\lambda_{\text{Th}}} e^{-\lambda_{\text{Th}}(t-t_0)} + \left( N_{\text{Pa}}(t_0) - \frac{\mathcal{A}_{\text{Th}}(t_0)}{\lambda_{\text{Pa}}-\lambda_{\text{Th}}} \right) e^{-\lambda_{\text{Pa}}(t-t_0)}\right) \\
							&= \frac{\mathcal{A}_{\text{Th}}(t_0) \lambda_{\text{Pa}}}{\lambda_{\text{Pa}}-\lambda_{\text{Th}}} e^{-\lambda_{\text{Th}}(t-t_0)} + \left( \mathcal{A}_{\text{Pa}}(t_0) - \frac{\mathcal{A}_{\text{Th}}(t_0) \lambda_{\text{Pa}}}{\lambda_{\text{Pa}}-\lambda_{\text{Th}}} \right) e^{-\lambda_{\text{Pa}}(t-t_0)} \\
							&= \mathcal{A}_{\text{Pa}}(t_0) e^{-\lambda_{\text{Pa}}(t-t_0)} +  \frac{\mathcal{A}_{\text{Th}}(t_0) \lambda_{\text{Pa}}}{\lambda_{\text{Pa}}-\lambda_{\text{Th}}} \left( e^{-\lambda_{\text{Th}}(t-t_0)} - e^{-\lambda_{\text{Pa}}(t-t_0)} \right) \\
\end{align*}
Using the numerical values for $\lambda_{\text{Th}}$, $\lambda_{\text{Pa}}$, $t$, and our answer from part (a) for the activities at $t=t_0=1.5$ hr, we find
\begin{dmath*}
\mathcal{A}_{\text{Pa}}(49.5\text{ hr}) = (0.006\text{ Ci}) e^{-(2.971\times10^{-7}\text{s}^{-1})(172800s)} +  \frac{(5.076\text{ Ci}) (2.971\times10^{-7}\text{s}^{-1})}{2.971\times10^{-7}\text{s}^{-1}-5.18\times10^{-4}\text{s}^{-1}} \left( e^{-(5.18\times10^{-4}\text{s}^{-1})(172800s)} - e^{-(2.971\times10^{-7}\text{s}^{-1})(172800s)} \right)
\end{dmath*}
$$\boxed{ \mathcal{A}_{\text{Pa}}(49.5\text{ hr}) = 0.008\text{ Ci} }$$

\item 

Note that $\lambda_{\text{U}} = \frac{\ln 2}{5.024^{12}\text{s}} = 1.380\times10^{-13}\text{s}^{-1}$. 

In just one day, the probability that any given $^{233}$Th nucleus survives is
$$ P = e^{-\lambda_{\text{Th}}t_d} $$
$$ P = e^{-(5.18\times10^{-4}\text{ s}^{-1})(86,400\text{ s})} \approx 10^{-20} $$
We can therefore assume that each thorium nucleus produced in the reactor has decayed into $^{233}$Pa by the end of the first day of storage. (There were only about $10^{15}$ thorium nuclei produced in total).
The probability that any one of these $^{233}$Pa atoms remains after one year---assumed to be 364 more days---is
$$ P = e^{-\lambda_{\text{Pa}}(t_y-t_d)} $$
$$ P = e^{-(2.971\times10^{-7}\text{ s}^{-1})(31449600\text{ s})} = 8.752\times10^{-5}  $$
While this probability indicates that some $^{233}$Pa nuclei will remain after the year of storage, they hardly make up a substantial fraction of the originally produced set of nuclei. We can assume that in one year, virtually every nucleus of $^{233}$Th has decayed at least into $^{233}$U.  

For uranium-233 on the other hand, the probability of survival for a nucleus over a year is
$$ P = e^{-\lambda_{\text{U}}t_y} $$
$$ P = e^{-(1.380\times10^{-13})(31536000\text{ s})} $$
$$ P = 0.999996 $$
We see that while nearly every nucleus decays from $^{233}$Th into $^{233}$Pa and then into $^{233}$U, almost no nuclei seem to decay from $^{233}$U in the single year.
We can then use our simple formula for the activity of the uranium sample
$$ \mathcal{A}_{\text{U}} = \lambda_{\text{U}} N_{\text{U}}$$

Using our rate of production, there are $2.0\times10^{11}\text{ s}^{-1}*(3600\text{ s/hr}\times 1.5 \text{hr}) \approx 1.08\times10^{15}$ produced in the irradiation period. We will assume this is also equal to the number of nuclei of $^{233}$U present at the end of the 1 year storage period: $N_U \approx 10^{15}$. We can finally calculate the activity, as 
$$ \mathcal{A}_{\text{U}} = (1.380\times10^{-13}\text{s}^{-1})(1.08\times10^{15}) .$$
This is ${149.04}\text{ Bq}$ or 
$$\boxed{ 4.028\times10^{-9}\text{ Ci} }$$
\end{enumerate}


\newpage
%%%%%%%%%%%%%%%%%%%%%%%%%%%%%%%%%% PROBLEM 2 %%%%%%%%%%%%%%%%%%%%%%%%%%%%%%%%%%
\section*{Problem 2}

Use the following masses for parts (a) and (b):

\begin{table}[htbp]
	\centering
	\begin{tabular}{|c|c|}
			\hline
			Nucleus		&	Atomic Mass \\
			\hline
			n 			&	 1.008665 u \\
			$^{1}$H		& 	 1.007825 u \\
			$^{2}$H 	&	 2.014102 u \\
			$^{56}$Fe	&   55.934939 u \\
			$^{98}$Y 	&   97.922203 u \\
			$^{135}$I	&  134.910048 u \\
			$^{235}$U	&  235.043924 u \\
			\hline
	\end{tabular}
	\label{tab:design-specs}
\end{table}
\begin{center}$1\text{u} \cdot c^{2} = 931.502$ MeV\end{center}
\-\\
\begin{enumerate}[a)]
\item Calculate the $Q$-value of the reaction:
$$ ^{235}\text{U}\, + \,n \-\rightarrow\- ^{135}\text{I}\, + \,^{98}\text{Y}\, + \,3n $$
\item Calculate the average binding energy per nucleon (in MeV) of $^{2}$H, $^{56}$Fe, and $^{235}$U.
\end{enumerate}



\section*{Problem 2 Solution}

\begin{enumerate}[a)]

\item 

The $Q$-value of a reaction is given by:$$Q=[m(\text{x})+m(\text{X})-m(\text{y})-m(\text{Y})]c^{2}.$$

\begin{spacing}{1.75}	
$^{235}\text{\normalfont{U}}+n \rightarrow  \-\ ^{135}\text{\normalfont{I}}+ \! ^{98}\text{\normalfont{Y}}+3n$\\
	\tab $Q = [m(^{235}\text{\normalfont{U}})+m(n)-m(^{135}\text{\normalfont{I}})-m(^{98}\text{\normalfont{Y}})-3m(n)]c^{2}$\\
	\tab $Q = [235.043924\text{u}+1.008665\text{u}-134.910048\text{u}-97.922203\text{u}-3(1.008665\text{u})]c^{2}$\\
	\tab $Q = 0.194343\text{u} \cdot c^{2}$\\
	\tab $\boxed{Q = 181.031 \text{ MeV}}$ 
\end{spacing}


\item 

The binding energy $B(Z,A)$ for any nuclei can be found approximately from the equation:$$M(Z,A) = Zm(^{1}\text{H})+(A-Z)m_{n}-B(Z,A)/c^{2},$$ where $M(Z,A),\; Zm(^{1}\text{H})$ and $m_{n}$ are experimentally calculated values. Per nucleon, the binding energy can be expressed as: $$B_{A}(Z,A) = [Zm(^{1}\text{H})+(A-Z)m_{n}-M(Z,A)]c^{2}/A.$$

\begin{spacing}{1.75}

\textbf{$^{2}$H}\\
$B_{A}(1,2) = [m(^{1}\text{H})+(2-1)m_{n}-M(1,2)]c^{2}/2$\\
$B_{A}(1,2) = [(1.007825\text{u})+(1.008665\text{u})-(2.014102\text{u})]c^{2}/2$\\
$B_{A}(1,2) = 0.001194\text{u} \cdot c^{2}$\\
$\boxed{B_{A}(1,2) = 1.112\text{ MeV}}$\\
\textbf{$^{56}$Fe}\\
$B_{A}(26,56) = [26m(^{1}\text{H})+(56-26)m_{n}-M(26,56)]c^{2}/56$\\
$B_{A}(26,56) = [26(1.007825\text{u})+30(1.008665\text{u})-(55.934939\text{u})]c^{2}/56$\\
$B_{A}(26,56) = 0.009437\text{u} \cdot c^{2}$\\
$\boxed{B_{Z}(26,56) = 8.791\text{ MeV}}$\\
\textbf{$^{235}$U}\\
$B_{A}(92,235) = [92m(^{1}\text{H})+(235-92)m_{n}-M(92,235)]c^{2}/235$\\
$B_{A}(92,235) = [92(1.007825\text{u})+143(1.008665\text{u})-(235.043924\text{u})]c^{2}/235$\\
$B_{A}(92,235) = 0.00814924 u\text{u} \cdot c^{2}$\\
$\boxed{B_{Z}(92,235) = 7.591\text{ MeV}}$
\end{spacing}

\end{enumerate}



\newpage
%%%%%%%%%%%%%%%%%%%%%%%%%%%%%%%%%% PROBLEM 3 %%%%%%%%%%%%%%%%%%%%%%%%%%%%%%%%%%
\section*{Problem 3}

\begin{enumerate}[a)]
\item Solve the first order differential equation
$$ \frac{dy}{dx} + 3y = 0 $$
\item Solve the second order differential equation ($A$ and $B$ are constants)
$$ \frac{d^2 y}{dx^2} - A^2y = B $$
The boundary condition is $y(\pm\frac{1}{A}) = 0$.
\end{enumerate}



\section*{Problem 3 Solution}
\begin{enumerate}[a)]

\item 

\begin{align*}
\frac{dy}{dx}	&= -3y \\
\frac{dy}{y} 	& = -3\,dx \\
\int\frac{dy}{y}&= -3 \int \,dx \\
\ln y			&= -3 x + C \\
y				&= e^{-3x + C} \\
\end{align*}
$$\boxed{y = Ce^{-3x}}$$

\item 

Solve the second order differential equation
$$ \frac{d^2 y}{dx^2} - A^2y = B $$
For the homogeneous equation, $\frac{d^2 y}{dx^2} - A^2y = 0$, try $y_c = X_1 e^{Ax} + X_2 e^{-Ax}$ as the complementary solution ($X_1$ and $X_2$ are constants). Then
$$ \frac{d^2 y}{dx^2} = X_1 A^2 e^{Ax} + X_2 A^2 e^{-Ax} $$
and
$$ (X_1 A^2 e^{Ax} + X_2 A^2 e^{-Ax}) - A^2(X_1 e^{Ax} + X_2 e^{-Ax}) = 0 $$
$$ A^2 (X_1 e^{Ax} + X_2 e^{-Ax}) - A^2 (X_1 e^{Ax} + X_2 e^{-Ax}) = 0 $$

This is \underline{true}.

For the inhomogeneous equation, $\frac{d^2 y}{dx^2} - A^2y = B$, $y_P = -B/A^2$ is the only particular solution satisfying the equation. The general solution is the sum of the complementary and particular solutions, $y = y_c + y_p$. 
$$ y = X_1 e^{Ax} + X_2 e^{-Ax} - \frac{B}{A^2} $$
Now we can solve for $X_1$ and $X_2$. 
$$ y(\frac{1}{A}) = X_1 e^{A(\frac{1}{A})} + X_2 e^{-A(\frac{1}{A})} - \frac{B}{A^2} = 0 $$
and
$$ y(-\frac{1}{A}) = X_1 e^{A(-\frac{1}{A})} + X_2 e^{A(\frac{1}{A})} - \frac{B}{A^2} = 0 $$
We can note that in these two equations, $X_1$ and $X_2$ can be interchanged freely, and so must be equal. We will say, $X_1 = X_2 = X$. Then, we have
$$ 0 = X e^{A(-\frac{1}{A})} + X e^{A(\frac{1}{A})} - \frac{B}{A^2} $$
$$ 0 = X (e^{-1} + e^{1}) - \frac{B}{A^2} $$
$$ X = \frac{B}{A^2(\frac{1}{e} + e)} $$
The we plug this $X$ into the final solution, which gives
$$ y = \frac{B}{A^2(\frac{1}{e} + e)} (e^{Ax} + e^{-Ax}) - \frac{B}{A^2} $$
\end{enumerate}
Note that $e^{Ax} + e^{-Ax}$ is similar in form to $\cosh(Ax)$ which we could have also used to solve this problem.



\newpage
%%%%%%%%%%%%%%%%%%%%%%%%%%%%%%%%%% PROBLEM 4 %%%%%%%%%%%%%%%%%%%%%%%%%%%%%%%%%%
\section*{Problem 4}

Classify the following cross section plots. They are, in no particular order:
\begin{enumerate}[(1)]
\item $^{155}$Gd absorption
\item $^{235}$U fission
\item $^{238}$U absorption
\item $^{238}$U fission
\item $^{239}$Pu fission
\end{enumerate}

(a) \includegraphics[width=8cm]{u238_fission.png}
(b) \includegraphics[width=8cm]{pu239_fission.png} \\
(c) \includegraphics[width=8cm]{gd155_absorption.png}
(d) \includegraphics[width=8cm]{u235_fission.png}\\
(e) \includegraphics[width=8cm]{u238_absorption.png}



\section*{Problem 4 Solution}


\tab\tab (a) \, 4 \, $^{238}$U fission (high fast region) 					\\
\tab\tab (b) \, 5 \, $^{239}$Pu fission (resonance peak)					\\
\tab\tab (c) \, 1 \, $^{155}$Gd absorption (consistently high absorption)	\\
\tab\tab (d) \, 2 \, $^{235}$U fission (high thermal fission cross section)	\\
\tab\tab (e) \, 3 \, $^{238}$U absorption (large absorptive resonances)



\end{document}