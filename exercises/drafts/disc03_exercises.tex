\documentclass{report}
% PACKAGES %
\usepackage[english]{} % Sets the language
\usepackage[margin=2cm]{geometry} % Sets the margin size
\usepackage{graphicx} % Enhanced package for including graphics/figures
\usepackage{float} % Allows figures and tables to be floats
\usepackage{amsmath} % Enhanced math package prepared by the American Mathematical Society
\usepackage{amssymb} % AMS symbols package
\usepackage{bm} % Allows you to use \bm{} to make any symbol bold
\usepackage{verbatim} % Allows you to include code snippets
\usepackage{setspace} % Allows you to change the spacing between lines at different points in the document
\usepackage{parskip} % Allows you alter the spacing between paragraphs
\usepackage{multicol} % Allows text division into multiple columns
\usepackage{units} % Allows fractions to be expressed diagonally instead of vertically
\usepackage{booktabs,multirow,multirow} % Gives extra table functionality
\usepackage{enumerate}
\newcommand{\tab}{\-\hspace{1.5cm}}

% Set path to figure image files
\graphicspath{ {fig/} }

\begin{document}

\begin{center}
\textbf{\large Nuclear Engineering 150 -- Discussion Section}\\ 
\textbf{Team Exercises \#3}
\end{center}

%%%%%%%%%%%%%%%%%%%%%%%%%%%%%%%%%% PROBLEM 1 %%%%%%%%%%%%%%%%%%%%%%%%%%%%%%%%%%
\section*{Problem 1}

A reactor is operating for a long time at some known power density $P_0$. Then, it instantaneously changes power to some power density $P_1$. One fission product of interest is $^{135}$Xe, though it has a neglible yield from the initial fission reaction. $^{135}$Xe precursors $^{135}$Te and $^{135}$I are produced with a combined yield of approximately 6\%, before decaying via $\beta^{-}$ decay to $^{135}$I and $^{135}$Xe respectively. Find the number density of $^{135}$Xe as a function of time after the power change. (Your solution may be left as variables)

\begin{table}[htbp]
	\centering
	\begin{tabular}{|c|c|c|}
			\hline
			Nucleus		&	Half-life 	& Thermal $\sigma_{\text{a}}$ \\
			\hline
			$^{135}$Te	&  $19.0$ s 	& $\sim 0$\\
			$^{135}$I	&  $6.6$ hr 	& $\sim 0$\\
			$^{135}$Xe	&  $9.2$ hr 	& $2.6 \times 10^6$ barns \\
			\hline
	\end{tabular}
	\label{tab:design-specs}
\end{table}



\newpage
%%%%%%%%%%%%%%%%%%%%%%%%%%%%%%%%%% PROBLEM 2 %%%%%%%%%%%%%%%%%%%%%%%%%%%%%%%%%%
\section*{Problem 2}

<<<<<<< HEAD
\textit{Text of problem 2}




=======
\begin{enumerate}[a)]
\item Find the excitation energy in $^{236}$U when a neutron with zero kinetic energy is absorbed by $^{235}$U. 
\item Find the excitation energy in $^{239}$U when a neutron with zero kinetic energy is absorbed by $^{238}$U. 
\item The activation energy for $^{236}$U is 6.2 MeV and the activation energy for $^{239}$U is 6.6 MeV. Will fission occur in each of these cases? Identify ${235}$U and ${238}$U as fissile or fissionable and explain.
\item A $^{238}$U nuclei absorbs a 2 MeV neutron and fissions into $^{132}$Sn, $^{106}$Mo, and a neutron. If the neutron is produced with 2.5\% of the total energy released in the reaction, does it have enough energy to fission another $^{238}$U atom?
\end{enumerate}

\begin{table}[htbp]
	\centering
	\begin{tabular}{|c|c|c|}
			\hline
			Nucleus		&	Mass 	\\
			\hline	
			$n$			&  amu \\
			$^{106}$Mo	&  amu \\
			$^{132}$Sn	&  amu \\
			$^{235}$U	&  amu \\
			$^{236}$U	&  amu \\	
			$^{238}$U	&  amu \\
			$^{239}$U	&  amu \\
			\hline
	\end{tabular}
	\label{tab:design-specs}
\end{table}
>>>>>>> exercises
\end{document}

