\documentclass{report}
% PACKAGES %
\usepackage[english]{} % Sets the language
\usepackage[margin=2cm]{geometry} % Sets the margin size
\usepackage{graphicx} % Enhanced package for including graphics/figures
\usepackage{float} % Allows figures and tables to be floats
\usepackage{amsmath} % Enhanced math package prepared by the American Mathematical Society
\usepackage{amssymb} % AMS symbols package
\usepackage{breqn} % Allows equation breaking over multiple lines
\usepackage{bm} % Allows you to use \bm{} to make any symbol bold
\usepackage{verbatim} % Allows you to include code snippets
\usepackage{setspace} % Allows you to change the spacing between lines at different points in the document
\usepackage{parskip} % Allows you alter the spacing between paragraphs
\usepackage{multicol} % Allows text division into multiple columns
\usepackage{units} % Allows fractions to be expressed diagonally instead of vertically
\usepackage{booktabs,multirow,multirow} % Gives extra table functionality
\usepackage{enumerate} % Allows you to use customizable bullets
\usepackage{fancyhdr} % Allows customizable headers and footers
\usepackage{tikz} % Allows the creation of diagrams
	\usetikzlibrary{shapes.geometric, arrows}
	\tikzstyle{isotope} = [rectangle, 
					  minimum width=2cm, 
					  minimum height=1.25cm,
					  text centered, 
					  draw=black, 
					 ]
	\tikzstyle{decay} = [rectangle, 
					      rounded corners,
					      minimum width=2cm, 
					      minimum height=1.5cm,
					      text centered, 
					      draw=white, 
					     ]
	\tikzstyle{placeholder} = [rectangle,
					      minimum width=2cm,
					      minimum height=1cm,
					      draw=white,
					      ]
	\tikzstyle{arrow} = [thick,->,>=stealth]
	\tikzstyle{farrow} = [ultra thick,->,>=stealth]
	
% Set path to figure image files
\graphicspath{ {fig/} }

% Set some custom shortcuts
\newcommand{\tab}{\-\hspace{1.5cm}}
\newcommand{\lap}{\nabla^2}
\newcommand{\p}{\partial}

% Set the header on the first page and copyright on all pages
\fancypagestyle{FirstPage}{
\chead{\textbf{Nuclear Engineering 150 -- Discussion Section}}
\rfoot{\small \copyright~2019 Mitchell Negus}
}
\fancypagestyle{EveryPage}{
\rfoot{\small \copyright~2019 Mitchell Negus}
}
\pagestyle{EveryPage}


\begin{document}

\begin{center}
\textbf{\large Nuclear Engineering 150 -- Discussion Section}\\ 
\textbf{Team Exercise Solutions \#4}
\end{center}

%%%%%%%%%%%%%%%%%%%%%%%%%%%%%%%%%% PROBLEM 1 %%%%%%%%%%%%%%%%%%%%%%%%%%%%%%%%%%
\section*{Problem 1}

A reactor contains 5\% (by atom) enriched uranium dioxide, which is 15\% of the entire core by volume.
\begin{enumerate}[a)]
\item Calculate the macroscopic cross section for this core if we were to treat it as a homogeneous volume. 
\item If the reactor were a cube with a side length of 4 m and a beam of $10^{15}$ thermal neutrons were incident on one face of the cube, how many neutrons would we expect to make it through to the other side uncollided?
\end{enumerate}

\begin{table}[htbp]
	\centering
	\begin{tabular}{|c|c|c|}
			\hline
			Nucleus		& Thermal $\sigma_{\text{t}}$ (b) & Mass (g/mol) \\
			\hline
			$^{1}$H		&  20.8						& 1.008			\\
			$^{16}$O	&  3.5 						& 15.995		\\
			$^{235}$U	&  607.5 					& 235.044		\\
			$^{238}$U	&  11.8						& 238.050		\\
			\hline
			Compound	& \multicolumn{2}{|c|}{$\rho \left(\text{g/cm}^3\right)$}          \\
			\hline
			H$_2$O		& \multicolumn{2}{|c|}{1.0}      \\
			UO$_2$		& \multicolumn{2}{|c|}{10.4}	\\
			\hline
	\end{tabular}
	\label{tab:design-specs}
\end{table}



\section*{Problem 1 Solution}

\begin{enumerate}[a)]

\item 

A macroscopic cross section for a homogeneous mixture of materials can be found by summing the cross sections of those components, weighted by their volume fractions: 
\begin{equation}
\label{gentotmacroXS}
\Sigma = \sum_{i} \frac{V_i}{V_t}\Sigma_i
\end{equation}
where $\Sigma_i$ is the macroscopic cross section of the $i^{\text{th}}$ material, $V_i$ is the volume of the $i^{\text{th}}$ material, and $V_t$ is the total volume. For this problem, we are told that the fraction $\frac{V_{\text{UO2}}}{V_t} = 0.15$ and $\frac{V_{\text{H2O}}}{V_t} = 0.85$. Using these in equation (\ref{gentotmacroXS}) the total macroscopic cross section of our reactor can be expressed just in terms of the macroscopic cross sections of UO$_2$ and H$_2$O.
\begin{equation}
\label{totmacroXS}
\Sigma = 0.15\Sigma_{\text{UO2}} + 0.85\Sigma_{\text{H2O}}
\end{equation}
Calculating these two macroscopic cross sections requires knowledge of each of the individual microscopic cross sections of the compounds, since the macroscopic cross section is defined as
\begin{equation}
\label{macroXSi}
\Sigma_i = n_i\sigma_i ,
\end{equation}
where $n_i$ and $\sigma_i$ are respectively the number density and total microscopic cross sections of the compounds. While neither of these quantities are given explicitly, we can calculate each from the information provided. 
\-\\

We can find a compound's number density by dividing the weight density by the mass of one molecule of the compound. 
$$ n_i = \frac{\rho_i}{m_i} $$
The densities for UO$_2$ and H$_2$O are provided, and we can find the mass of a molecule by dividing the molar mass of the material by Avogadro's number, $N_A = 6.022\times10^{23}\text{ molecules/mol}$.
$$ m_i = \frac{M_i}{N_A}, $$
and so
$$ n_i = \frac{\rho_i N_A}{M_i} .$$
For our two compounds, UO$_2$ and H$_2$O these number densities are
$$ n_{\text{UO}_2} = \frac{\rho_{\text{UO}_2} N_A}{M_{\text{UO}_2}} \qquad\text{and}\qquad n_{\text{H}_2\text{O}} = \frac{\rho_{\text{H}_2\text{O}} N_A}{M_{\text{H}_2\text{O}}} .$$
Before moving on to finding the microscopic cross sections, we can note that the molar mass of the compounds can be broken into the sum of the molar masses of their constituents according to their fraction in the compound.
$$ n_{\text{UO}_2} = \frac{\rho_{\text{UO}_2} N_A}{f_{\text{U}5}M_{\text{U235}} + f_{\text{U}8}M_{\text{U238}} + 2M_{\text{O}}} \qquad\text{and}\qquad n_{\text{H}_2\text{O}} = \frac{\rho_{\text{H}_2\text{O}} N_A}{2M_{\text{H}} + M_{\text{O}}} .$$
\-\\

Next, we look at microscopic cross sections. The microscopic cross section of a compound is given by the sum of the microscopic cross sections of it's elemental components. The complete microscopic cross section for water can be found using this fact.
$$ \sigma_{\text{H2O}} = 2\sigma_{\text{H}} + \sigma_{\text{O}} $$
The microscopic cross section for UO$_2$ can be found similarly, however the calculation is slightly more complicated due to its enrichment in $^{235}$U. First, we find the average microscopic cross section for uranium by weighting the cross sections of $^{235}$U and $^{238}$U by their abundance.
$$ \sigma_{\text{U}} = f_{\text{U}5} \sigma_{\text{U}5} + f_{\text{U}8} \sigma_{\text{U}8} $$
where $f_{i}$ is the atomic fraction of material $i$ in the compound (for uranium in this case, this fraction is just the enrichment in atom \%). Then, we can complete the process exactly as we did for H$_2$O,
\begin{align*}
\sigma_{\text{UO}_2}	&= \sigma_{\text{U}} + 2\sigma_{\text{O}} \\
						&= f_{\text{U}5} \sigma_{\text{U}5} + f_{\text{U}8} \sigma_{\text{U}8}  + 2\sigma_{\text{O}}. 
\end{align*}
\-\\

Combining these number densities and macroscopic cross sections, we can expand and rewrite the macroscopic cross sections described by equation (\ref{macroXSi}) as
%$$ \Sigma_{\text{UO}_2} = \frac{\rho_{\text{UO}_2} \left( f_{\text{U}5} \sigma_{\text{U}5} + f_{\text{U}8} \sigma_{\text{U}8}  + 2\sigma_{\text{O}}\right) N_A}{f_{\text{U}5}M_{\text{U235}} + f_{\text{U}8}M_{\text{U238}} + 2M_{\text{O}}} \qquad\text{and}\qquad \Sigma_{\text{H}_2\text{O}} = \frac{\rho_{\text{H}_2\text{O}} \left(2\sigma_{\text{H}} + \sigma_{\text{O}}\right) N_A}{2M_{\text{H}} + M_\text{O}} $$
Now we can substitute the given values. 
$$ \Sigma_{\text{UO}_2} = \frac{\left(10.4\text{ g/cm}^3\right) \left(0.05(607.5\text{ b}) + 0.95(11.8\text{ b})  + 2(3.5\text{ b})\right)\left(6.022\times10^{23}\text{ mol}^{-1}\right)}{0.05(235.044\text{ g/mol}) + 0.95(238.050\text{ g/mol}) + 2(15.995\text{ g/mol})} $$
$$ \Sigma_{\text{UO}_2} = 112.7\text{ m}^{-1} $$

$$ \Sigma_{\text{H}_2\text{O}} = \frac{\left(1.0\text{ g/cm}^3\right) \left(2(20.8\text{ b}) + 3.5\text{ b}\right) \left(6.022\times10^{23}\text{ mol}^{-1}\right)}{2(1.008\text{ g/mol}) + 15.995\text{ g/mol}} $$
$$ \Sigma_{\text{H}_2\text{O}} = 150.8\text{ m}^{-1} $$

and plug these into equation (\ref{totmacroXS}) for the total cross section of the core.
$$ \Sigma = 0.15(112.7\text{ m}^{-1}) + 0.85(150.8\text{ m}^{-1})
 $$
$$\boxed{ \Sigma = 145.085\text{ m}^{-1} }$$

\item 

Neutron attentuation follows a decaying exponential according to the equation
$$ N = N(0) \, e^{-\Sigma x} $$
where $N$ is the number of neutrons (you may be more familiar with this equation in terms of intensity; we have just eliminated the area and rate components from both sides).

We let $N(0)$ be our initial number of incident particles, $10^{15}$, $\Sigma$ be the macroscopic cross section calculated in part (a), and $x$ be the distance traveled by the neutrons, 4 m.  
$$ N = 10^{15} e^{(-145.085\text{ m}^{-1})(4\text{ m})} $$
This works out to be about $9.153\times10^{-238}$, or \underline{zero particles}.
$$ N = 0 $$
Note that this number only indicates the quantity of particles which make it through the core uncollided. Since many of these interactions are scattering collisions, and not absorptive events, it is quite possible that some neutrons will eventually make it through the reactor. A more complicated derivation would be required in that case.
\end{enumerate}



\newpage
%%%%%%%%%%%%%%%%%%%%%%%%%%%%%%%%%% PROBLEM 2 %%%%%%%%%%%%%%%%%%%%%%%%%%%%%%%%%%
\section*{Problem 2}

A monoenergetic neutron beam with an intensity of $2 \times 10^{12}\text{ neutrons/(cm}^2{\cdot}\text{s)}$ is incident on an unknown shielding material and has a beam spot of 5 cm$^2$. The shielding material has a thickness of 10 cm.
\begin{enumerate}[a)]
\item On average, $3.0\times10^9$ neutrons/s make it through the shield uncollided. What is the macroscopic cross section of the shield material?
\item What is the mean free path of a neutron in the shielding material?
\item If a single beam pulse is 10 $\mu$s, how many collisions are expected to take place in the shielding material?
\end{enumerate}



\section*{Problem 2 Solution}

\begin{enumerate}[a)]

\item 

The transmitted uncollided intensity of a neutron beam through a material with macroscopic cross section $\Sigma$ is given by
$$ I = I(0) \, e^{-\Sigma x} .$$
Solving for $\Sigma$, we find 
$$ \Sigma = \frac{1}{x}\ln\left(\frac{I(0)}{I}\right) $$
Using the values provided (and considering the intensity as per the target, rather than per cm$^2$), we have
$$ \Sigma = \frac{1}{10\text{ cm}}\ln\left(\frac{2\times10^{12}\text{ neutrons/(cm}^2{\cdot}\text{s)}\left( 5\text{ cm}^2 \right)}{3.0\times10^9\text{ neutrons/s}}\right) $$
$$ \Sigma = \frac{1}{10\text{ cm}}\ln\left(3.33\times10^{3}\right) $$
$$\boxed{ \Sigma = 0.811\text{ cm}^{-1} }$$

\item 

The mean free path of a particle is defined as $\lambda \equiv \frac{1}{\Sigma}$. 
$$\boxed{ \lambda = 1.233\text{ cm} }$$

\item 

We are told that the beam pulse is 10 $\mu$s, so when considered in conjunction with the known intensity of the neutron beam, we can find the total number of particles produced by the beam per area. We define the neutron fluence, $\Phi$, as the number of neutrons per cm$^2$, calculated as
$$ \Phi = It = (2\times10^{12}\text{ neutrons/(cm}^2\cdot\text{s)})(10^{-5}\text{ s}) = 2\times10^7\text{ neutrons/cm}^2 $$
The number of interactions, $R$, is 
$$ R = \Phi \Sigma V $$
where $\Phi$ is the incident neutron fluence in $\left[\frac{\text{neutrons}}{\text{cm}^2}\right]$, $\Sigma$ is the macroscopic cross section of the shielding material in $\left[\frac{1}{\text{cm}}\right]$, and $V$ is the volume of the shield in $\left[{\text{cm}^3}\right]$. 

We have already found $\Phi$, we calculated $\Sigma$ in part (a), and we can determine $V$ by multiplying the area of the beam spot by the thickness of the target.
$$ V = 5\text{ cm}^2 \times 10\text{ cm} = 50\text{ cm}^3 $$
The total number of collisions is then
$$ R = (2\times10^7\text{ cm}^{-2})(0.811\text{ cm}^{-1})(50\text{ cm}^3) $$
$$\boxed{ R = 8.11\times10^8\text{ collisions} }$$
\end{enumerate}


\end{document}

Potential updates for problem 1: 
-ask how far it takes for no beam neutrons to be remaining
-ask how many mean-free paths this is




