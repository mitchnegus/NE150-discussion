\section*{Problem 1 Solution}

For all parts of this problem, let $R$ be the rate of neutron capture by 1 g of $^{232}$Th, $2.0 \times 10^{11} \;\text{s}^{-1}$.\\

\begin{enumerate}[a)]

\item

First, convert all half-lives and irradiation times to seconds for consistency.\\ 
\-\\
\tab $\lambda_{\text{Th}} = \frac{\ln{2}}{1338\text{s}} = 5.18\times10^{-4}\text{s}^{-1}$; \tab $\lambda_{\text{Pa}}=\frac{\ln{2}}{2332800s}=2.971\times10^{-7}\text{s}^{-1}$; \tab 1.5 hr = 5400 s\\

\textbf{Thorium-233}\\
\-\\
The rate of change of the quantity of $^{233}$Th, $\frac{dN_{\text{th}}}{dt}$ is given by the production rate of $^{233}$Th, $R$, minus the decay rate (activity) of $^{233}$Th, $\lambda_{\text{Th}}N_{\text{Th}}$.
$$\frac{dN_{\text{Th}}}{dt} = R - \lambda_{\text{Th}}N_{\text{Th}}$$
We solve the differential equation, manipulating the equation so that the left side is only dependent on $N_{\text{Th}}$ and the right side is only dependent on $dt$. Then we integrate:
$$\int{\frac{dN_{\text{Th}}}{R-\lambda_{\text{Th}}N_{\text{Th}}}} = \int{dt}$$
$$\frac{-1}{\lambda_{\text{Th}}}[\ln(R-\lambda_{\text{Th}}N_{\text{Th}})] = t + C,\quad C=\text{const.}$$
$$R-\lambda_{\text{Th}}N_{\text{Th}} = e^{-\lambda_{\text{Th}}t -\lambda_{\text{Th}}C} = e^{-\lambda_{\text{Th}}t} e^{-\lambda_{\text{Th}}C}$$
We note that since $C$ is an arbitrary constant and $\lambda_{\text{Th}}$ is fixed, we could also say $e^{-\lambda_{\text{Th}}C}$ is an arbitrary constant, and just call it $C$ instead.
$$ R-\lambda_{\text{Th}}N_{\text{Th}} = Ce^{-\lambda_{\text{Th}}t} $$
We solve for $N_{\text{Th}}$ (explicitly including $N_{\text{Th}}$'s dependence on $t$), and get
$$ N_{\text{Th}}(t) = \frac{R - Ce^{-\lambda_{\text{Th}}t}}{\lambda_{\text{Th}}} .$$
At $t=0,\; N_{\text{Th}}(0) = \frac{R - C}{\lambda_{\text{Th}}}=0,$ since no $^{233}$\text{Th} has been formed. We find $C= R$, and use this in the general equation:
$$ N_{\text{Th}}(t) = R\frac{1 - e^{-\lambda_{\text{Th}}t}}{\lambda_{\text{Th}}}. $$
With this function of $N_{\text{Th}}$, we can determine the activity as a function of time, knowing that
$$ \mathcal{A}_{\text{Th}}(t) = \lambda_{\text{Th}}N_{\text{Th}}(t). $$
Substituting, we find
$$ \mathcal{A}_{\text{Th}}(t) = R(1-e^{-\lambda_{\text{Th}}t}) .$$
Using the numerical values for $R$, $\lambda_{\text{Th}}$, and $t$,
$$ \mathcal{A}_{\text{Th}}(1.5\text{ hr}) =(2.0\times10^{11}s^{-1})(1-e^{(-5.18\times10^{-4}\text{s}^{-1})(5400s)}) $$
$$ \mathcal{A}_{\text{Th}}(1.5\text{ hr}) = 1.878\times10^{11}\text{ Bq} .$$
Finally, we convert this to Curies,
$$\boxed{ \mathcal{A}_{\text{Th}}(1.5\text{ hr}) = 5.076\text{ Ci} }.$$

\textbf{Protactinium-233}\\
\-\\
We follow a similar procedure for $^{233}$Pa, noting that the production rate of $^{233}$Pa is just the activity of $^{233}$Th as it decays into $^{233}$Pa, $\mathcal{A}_{\text{Th}}$.
$$\frac{dN_{\text{Pa}}}{dt} = \mathcal{A}_{\text{Th}} - \lambda_{\text{Pa}}N_{\text{Pa}}$$
From above, we can substitute our function for $\mathcal{A}_{\text{Th}}(t)$,
$$\frac{dN_{\text{Pa}}}{dt} = R(1-e^{-\lambda_{\text{Th}}t}) - \lambda_{\text{Pa}}N_{\text{Pa}}$$
Since we cannot separate both sides to be dependent only on a single differential, we must try a different method of integration. We will use integrating factors. Still, we start in a similar fashion: collecting the terms dependent on $N_{\text{Pa}}$ on the same side.
$$ \frac{dN_{\text{Pa}}}{dt}+\lambda_{\text{Pa}}N_{\text{Pa}} = R(1-e^{-\lambda_{\text{Th}}t}) $$
The method of integrating factors suggests that we multiply both sides by an arbitrary exponential. We will use $e^{\lambda_{\text{Pa}}t}$.
$$ e^{\lambda_{\text{Pa}}t}\frac{dN_{\text{Pa}}}{dt} + \lambda_{\text{Pa}}e^{\lambda_{\text{Pa}}t}N_{\text{Pa}} = e^{\lambda_{\text{Pa}}t}R(1-e^{-\lambda_{\text{Th}}t}) $$
We can now observe that the left side of the equation appears to be the result of the product rule when the time derivative of $e^{\lambda_{\text{Pa}}t}N_{\text{Pa}}$ is found. We can then write the equation as
$$\frac{d}{dt}(e^{\lambda_{\text{Pa}}t}N_{\text{Pa}}) = e^{\lambda_{\text{Pa}}t}R(1-e^{-\lambda_{\text{Th}}t})$$
Moving the $dt$ term to the right side of the equation and using the distributive property, we have
$$ d(e^{\lambda_{\text{Pa}}t}N_{\text{Pa}}) = R(e^{\lambda_{\text{Pa}}t}-e^{\lambda_{\text{Pa}}t}e^{-\lambda_{\text{Th}}t})dt $$
or more simply (by exploiting properties of exponents)
$$ d(e^{\lambda_{\text{Pa}}t}N_{\text{Pa}}) = R(e^{\lambda_{\text{Pa}}t}-e^{\lambda_{\text{Pa}}t-\lambda_{\text{Th}}t})dt .$$
We integrate both sides,
$$ \int{d\left(e^{\lambda_{\text{Pa}}t}N_{\text{Pa}}\right)} = \int{ R(e^{\lambda_{\text{Pa}}t}-e^{\lambda_{\text{Pa}}t-\lambda_{\text{Th}}t})dt}, $$
separate the integral on the right side,
$$ \int{d\left(e^{\lambda_{\text{Pa}}t}N_{\text{Pa}}\right)} = R\int{ e^{\lambda_{\text{Pa}}t}dt}-R\int{e^{(\lambda_{\text{Pa}}-\lambda_{\text{Th}})t}dt} $$
and find
$$e^{\lambda_{\text{Pa}}t}N_{\text{Pa}} = \frac{R}{\lambda_{\text{Pa}}}e^{\lambda_{\text{Pa}}t}-\frac{R}{\lambda_{\text{Pa}}-\lambda_{\text{Th}}}e^{(\lambda_{\text{Pa}}-\lambda_{\text{Th}})t} +\;C,\; C=\text{const.}$$
Now we factor out the integrating factor back out from both sides and note explicitly the time dependence of $N_{\text{Pa}}$,
$$N_{\text{Pa}}(t) = \frac{R}{\lambda_{\text{Pa}}}-\frac{R}{\lambda_{\text{Pa}}-\lambda_{\text{Th}}}e^{-\lambda_{\text{Th}}t}+Ce^{-\lambda_{\text{Pa}}t}$$
At $t=0$,
$$ N_{\text{Pa}}(0) = \frac{R}{\lambda_{\text{Pa}}}-\frac{R}{\lambda_{\text{Pa}}-\lambda_{\text{Th}}}+C = 0, $$
 since no $^{233}$\text{Pa} has been formed. Solving for $C$, we find $C = \frac{R}{\lambda_{\text{Pa}}-\lambda_{\text{Th}}}-\frac{R}{\lambda_{\text{Pa}}}$. We plug this back into our equation above, and have the solution for $N_{\text{Pa}}(t)$:
$$N_{\text{Pa}}(t) = \frac{R}{\lambda_{\text{Pa}}}-\frac{R}{\lambda_{\text{Pa}}-\lambda_{\text{Th}}}e^{-\lambda_{\text{Th}}t}+\left( \frac{R}{\lambda_{\text{Pa}}-\lambda_{\text{Th}}}-\frac{R}{\lambda_{\text{Pa}}} \right)e^{-\lambda_{\text{Pa}}t}$$
and simplifying
$$ N_{\text{Pa}}(t) = \frac{R}{\lambda_{\text{Pa}}}(1-e^{\lambda_{\text{Pa}}t})+ \left(\frac{R}{\lambda_{\text{Pa}}-\lambda_{\text{Th}}}\right)(e^{-\lambda_{\text{Pa}}t}-e^{-\lambda_{\text{Th}}t}) .$$
With this function of $N_{\text{Pa}}$, we can determine the activity as a function of time, knowing that
$$ \mathcal{A}_{\text{Pa}}(t) = \lambda_{\text{Pa}}N_{\text{Pa}}(t) .$$
Substituting, we find
$$ \mathcal{A}_{\text{Pa}}(t) = R(1-e^{-\lambda_{\text{Pa}}t})+ \left(\frac{R \, \lambda_{\text{Pa}}}{\lambda_{\text{Pa}}-\lambda_{\text{Th}}}\right)(e^{-\lambda_{\text{Pa}}t}-e^{-\lambda_{\text{Th}}t}) .$$
Using the numerical values for $R$, $\lambda_{\text{Th}}$, $\lambda_{\text{Pa}}$, and $t$,
\begin{dmath*}
\mathcal{A}_{\text{Pa}}(1.5\text{ hr}) = (2.0\times10^{11}\text{ s}^{-1})\left(1-e^{(-2.971\times10^{-7}\text{s}^{-1})(5400\text{s})}\right)+ \left(\frac{2.0\times10^{11}\text{ s}^{-1}(2.791\times10^{-7}\text{s}^{-1})}{2.971\times10^{-7}\text{s}^{-1}-5.18\times10^{-4}\text{s}^{-1}}\right)\left(e^{(-2.971\times10^{-7}\text{s}^{-1})(5400\text{s})}-e^{(-5.18\times10^{-4}\text{s}^{-1})(5400\text{s})}\right)
\end{dmath*}
$$ \mathcal{A}_{\text{Pa}}(1.5\text{ hr}) = 2.195 \times 10^{8}\text{ Bq} $$
Finally, we convert this to Curies, 
$$ \boxed{\mathcal{A}_{\text{Pa}}(1.5\text{ hr}) = 0.006\text{ Ci}}. $$

\item

Let's say that the 1.5 hour mark is now given by $t=t_0=1.5\text{ hr}$. We also note that 48 hours = 172,800 seconds.

\textbf{Thorium-233}\\
\-\\
Without irradiation, the rate of change of the quantity of $^{233}$Th is now just the decay rate. 
$$ \frac{dN_{\text{Th}}}{dt} = -\lambda_{\text{Th}}N_{\text{Th}} $$
We separate the equation and integrate, arriving at the standard exponential decay formula, now including the explicit time dependence.
$$ \int_{N_{\text{Th}}(t_0)}^{N_{\text{Th}}(t)} \frac{-dN_{\text{Th}}'}{\lambda_{\text{Th}}N_{\text{Th}}'} = \int_{t_0}^{t} dt' $$
$$ \frac{-1}{\lambda_{\text{Th}}}\left[ \ln{N_{\text{Th}}} \right]_{N_{\text{Th}(t_0)}}^{N_{\text{Th}}(t)} = \left[t'\right]_{t_0}^{t} $$
$$ \ln\frac{N_{\text{Th}}(t)}{N_{\text{Th}}(t_0)}  = -\lambda_{\text{Th}}(t-t_0) $$
$$ \frac{N_{\text{Th}}(t)}{N_{\text{Th}}(t_0)}  = e^{-\lambda_{\text{Th}}(t-t_0)} $$
and we have 
$$ N_{\text{Th}}(t) = N_{\text{Th}}(t_0) e^{-\lambda_{\text{Th}}\left(t-t_0\right)} $$
Given the definition of activity as $\mathcal{A} = \lambda N$ and noting $\lambda_{\text{Th}} N_{\text{Th}}(t_0) = \mathcal{A}_{\text{Th}}(t_0)$, we can write the activity of $^{233}$Th as
\begin{align*}
\mathcal{A}_{\text{Th}}(t)	&= \lambda_{\text{Th}} N_{\text{Th}}(t) \\
							&= \lambda_{\text{Th}} N_{\text{Th}}(t_0) e^{-\lambda_{\text{Th}}(t-t_0)} \\ 
							&= \mathcal{A}_{\text{Th}}(t_0) e^{-\lambda_{\text{Th}}(t-t_0)}
\end{align*}
Using the numerical values for $\lambda_{\text{Th}}$, $t$, and our answer from part (a) for the activity at $t_0=1.5$ hr, we find
$$ \mathcal{A}_{\text{Th}}(49.5\text{ hr}) = (5.076 \text{ Ci})e^{-5.18\times10^{-4}\text{s}^{-1} (172800\text{s})} $$
$$ \boxed{\mathcal{A}_{\text{Th}}(49.5\text{ hr}) = 6.787\times10^{-39}\text{ Ci}} $$

{\small Note: we could also have assumed that since 48 hours is many (more than 100) times longer than the half-life of $^{233}$Th, that the activity would be approximately zero.}
\-\\
\-\\

\textbf{Protactinium-233}\\
\-\\
We follow the example in part (a) for $^{233}$Pa, again using the activity of $^{233}$Th as the production rate of $^{233}$Pa.
$$ \frac{dN_{\text{Pa}}}{dt} = \mathcal{A}_{\text{Th}} - \lambda_{\text{Pa}}N_{\text{Pa}}. $$
We collect terms dependent on $N_{\text{Pa}}$ on one side,
$$ \frac{dN_{\text{Pa}}}{dt} + \lambda_{\text{Pa}}N_{\text{Pa}} = \mathcal{A}_{\text{Th}}(t_0) e^{-\lambda_{\text{Th}}(t-t_0)}, $$
multiply both sides by arbitrary exponential $e^{\lambda_{\text{Pa}}t}$,
$$ e^{\lambda_{\text{Pa}}t}\frac{dN_{\text{Pa}}}{dt} + \lambda_{\text{Pa}}e^{\lambda_{\text{Pa}}t}N_{\text{Pa}} = e^{\lambda_{\text{Pa}}t}\mathcal{A}_{\text{Th}}(t_0) e^{-\lambda_{\text{Th}}(t-t_0)}, $$
note that the left side is the result of the product rule when $\frac{d}{dt}$ is taken on $e^{\lambda_{\text{Pa}}t}N_{\text{Pa}}$
$$ \frac{d}{dt} (e^{\lambda_{\text{Pa}}t}N_{\text{Pa}}) = e^{\lambda_{\text{Pa}}t}\mathcal{A}_{\text{Th}}(t_0) e^{-\lambda_{\text{Th}}(t-t_0)}, $$
multiply by the differential, $dt$,
$$ d(e^{\lambda_{\text{Pa}}t}N_{\text{Pa}}) = e^{\lambda_{\text{Pa}}t}\mathcal{A}_{\text{Th}}(t_0) e^{-\lambda_{\text{Th}}(t-t_0)} dt ,$$
and rearrange,
$$ d(e^{\lambda_{\text{Pa}}t}N_{\text{Pa}}) = \mathcal{A}_{\text{Th}}(t_0) e^{(\lambda_{\text{Pa}} - \lambda_{\text{Th}})t + \lambda_{\text{Th}}t_0} dt .$$
Then we integrate,
$$ \int_{e^{\lambda_{\text{Pa}}t_0}N_{\text{Pa}}(t_0)}^{e^{\lambda_{\text{Pa}}t}N_{\text{Pa}}(t)} d(e^{\lambda_{\text{Pa}}t}N_{\text{Pa}}') = \mathcal{A}_{\text{Th}}(t_0)e^{\lambda_{\text{Th}}t_0} \int_{t_0}^{t} e^{(\lambda_{\text{Pa}} - \lambda_{\text{Th}})t'} dt' ,$$
and find,
$$ \left[e^{\lambda_{\text{Pa}}t}N_{\text{Pa}}'\right]_{e^{\lambda_{\text{Pa}}t_0}N_{\text{Pa}}(t_0)}^{e^{\lambda_{\text{Pa}}t}N_{\text{Pa}}(t)} = \frac{\mathcal{A}_{\text{Th}}(t_0)e^{\lambda_{\text{Th}}t_0}}{\lambda_{\text{Pa}}-\lambda_{\text{Th}}} \left[ e^{(\lambda_{\text{Pa}}-\lambda_{\text{Th}})t'}\right]_{t_0}^{t} $$
$$ e^{\lambda_{\text{Pa}}t}N_{\text{Pa}}(t) - e^{\lambda_{\text{Pa}}t_0}N_{\text{Pa}}(t_0) = \frac{\mathcal{A}_{\text{Th}}(t_0)e^{\lambda_{\text{Th}}t_0}}{\lambda_{\text{Pa}}-\lambda_{\text{Th}}} \left[ e^{(\lambda_{\text{Pa}}-\lambda_{\text{Th}})t} - e^{(\lambda_{\text{Pa}}-\lambda_{\text{Th}})t_0} \right] $$
Factoring out the integrating factor, 
$$ N_{\text{Pa}}(t) - e^{-\lambda_{\text{Pa}}(t - t_0)}N_{\text{Pa}}(t_0) = \frac{\mathcal{A}_{\text{Th}}(t_0)e^{\lambda_{\text{Th}}t_0}}{\lambda_{\text{Pa}}-\lambda_{\text{Th}}} \left[ e^{-\lambda_{\text{Th}}t} - e^{-\lambda_{\text{Pa}}t}e^{(\lambda_{\text{Pa}}-\lambda_{\text{Th}})t_0} \right] $$
and solve for $ N_{\text{Pa}}(t)$,
$$ N_{\text{Pa}}(t) = \frac{\mathcal{A}_{\text{Th}}(t_0)e^{\lambda_{\text{Th}}t_0}}{\lambda_{\text{Pa}}-\lambda_{\text{Th}}} \left[ e^{-\lambda_{\text{Th}}t} - e^{-\lambda_{\text{Pa}}t}e^{(\lambda_{\text{Pa}}-\lambda_{\text{Th}})t_0} \right] + e^{-\lambda_{\text{Pa}}(t-t_0)}N_{\text{Pa}}(t_0) .$$
Rearranging,
$$ N_{\text{Pa}}(t) = \frac{\mathcal{A}_{\text{Th}}(t_0)}{\lambda_{\text{Pa}}-\lambda_{\text{Th}}} \left[ e^{-\lambda_{\text{Th}}(t - t_0)} - e^{-\lambda_{\text{Pa}}(t-t_0)} \right] + e^{-\lambda_{\text{Pa}}(t - t_0)}N_{\text{Pa}}(t_0) .$$
and so
$$ N_{\text{Pa}}(t) = \frac{\mathcal{A}_{\text{Th}}(t_0)}{\lambda_{\text{Pa}}-\lambda_{\text{Th}}} e^{-\lambda_{\text{Th}}(t-t_0)} + \left( N_{\text{Pa}}(t_0) - \frac{\mathcal{A}_{\text{Th}}(t_0)}{\lambda_{\text{Pa}}-\lambda_{\text{Th}}} \right) e^{-\lambda_{\text{Pa}}(t-t_0)} $$
Given the definition of activity as $\mathcal{A} = \lambda N$ and noting $\lambda_{\text{Pa}} N_{\text{Pa}}(t_0) = \mathcal{A}_{\text{Pa}}(t_0)$, we can write the activity of $^{233}$Pa as
\begin{align*}
\mathcal{A}_{\text{Pa}}(t)	&= \lambda_{\text{Pa}} N_{\text{Pa}}(t) \\
							&=\lambda_{\text{Pa}} \left(\frac{\mathcal{A}_{\text{Th}}(t_0)}{\lambda_{\text{Pa}}-\lambda_{\text{Th}}} e^{-\lambda_{\text{Th}}(t-t_0)} + \left( N_{\text{Pa}}(t_0) - \frac{\mathcal{A}_{\text{Th}}(t_0)}{\lambda_{\text{Pa}}-\lambda_{\text{Th}}} \right) e^{-\lambda_{\text{Pa}}(t-t_0)}\right) \\
							&= \frac{\mathcal{A}_{\text{Th}}(t_0) \lambda_{\text{Pa}}}{\lambda_{\text{Pa}}-\lambda_{\text{Th}}} e^{-\lambda_{\text{Th}}(t-t_0)} + \left( \mathcal{A}_{\text{Pa}}(t_0) - \frac{\mathcal{A}_{\text{Th}}(t_0) \lambda_{\text{Pa}}}{\lambda_{\text{Pa}}-\lambda_{\text{Th}}} \right) e^{-\lambda_{\text{Pa}}(t-t_0)} \\
							&= \mathcal{A}_{\text{Pa}}(t_0) e^{-\lambda_{\text{Pa}}(t-t_0)} +  \frac{\mathcal{A}_{\text{Th}}(t_0) \lambda_{\text{Pa}}}{\lambda_{\text{Pa}}-\lambda_{\text{Th}}} \left( e^{-\lambda_{\text{Th}}(t-t_0)} - e^{-\lambda_{\text{Pa}}(t-t_0)} \right) \\
\end{align*}
Using the numerical values for $\lambda_{\text{Th}}$, $\lambda_{\text{Pa}}$, $t$, and our answer from part (a) for the activities at $t=t_0=1.5$ hr, we find
\begin{dmath*}
\mathcal{A}_{\text{Pa}}(49.5\text{ hr}) = (0.006\text{ Ci}) e^{-(2.971\times10^{-7}\text{s}^{-1})(172800s)} +  \frac{(5.076\text{ Ci}) (2.971\times10^{-7}\text{s}^{-1})}{2.971\times10^{-7}\text{s}^{-1}-5.18\times10^{-4}\text{s}^{-1}} \left( e^{-(5.18\times10^{-4}\text{s}^{-1})(172800s)} - e^{-(2.971\times10^{-7}\text{s}^{-1})(172800s)} \right)
\end{dmath*}
$$\boxed{ \mathcal{A}_{\text{Pa}}(49.5\text{ hr}) = 0.008\text{ Ci} }$$

\item 

Note that $\lambda_{\text{U}} = \frac{\ln 2}{5.024^{12}\text{s}} = 1.380\times10^{-13}\text{s}^{-1}$. 

In just one day, the probability that any given $^{233}$Th nucleus survives is
$$ P = e^{-\lambda_{\text{Th}}t_d} $$
$$ P = e^{-(5.18\times10^{-4}\text{ s}^{-1})(86,400\text{ s})} \approx 10^{-20} $$
We can therefore assume that each thorium nucleus produced in the reactor has decayed into $^{233}$Pa by the end of the first day of storage. (There were only about $10^{15}$ thorium nuclei produced in total).
The probability that any one of these $^{233}$Pa atoms remains after one year---assumed to be 364 more days---is
$$ P = e^{-\lambda_{\text{Pa}}(t_y-t_d)} $$
$$ P = e^{-(2.971\times10^{-7}\text{ s}^{-1})(31449600\text{ s})} = 8.752\times10^{-5}  $$
While this probability indicates that some $^{233}$Pa nuclei will remain after the year of storage, they hardly make up a substantial fraction of the originally produced set of nuclei. We can assume that in one year, virtually every nucleus of $^{233}$Th has decayed at least into $^{233}$U.  

For uranium-233 on the other hand, the probability of survival for a nucleus over a year is
$$ P = e^{-\lambda_{\text{U}}t_y} $$
$$ P = e^{-(1.380\times10^{-13})(31536000\text{ s})} $$
$$ P = 0.999996 $$
We see that while nearly every nucleus decays from $^{233}$Th into $^{233}$Pa and then into $^{233}$U, almost no nuclei seem to decay from $^{233}$U in the single year.
We can then use our simple formula for the activity of the uranium sample
$$ \mathcal{A}_{\text{U}} = \lambda_{\text{U}} N_{\text{U}}$$

Using our rate of production, there are $2.0\times10^{11}\text{ s}^{-1}*(3600\text{ s/hr}\times 1.5 \text{hr}) \approx 1.08\times10^{15}$ produced in the irradiation period. We will assume this is also equal to the number of nuclei of $^{233}$U present at the end of the 1 year storage period: $N_U \approx 10^{15}$. We can finally calculate the activity, as 
$$ \mathcal{A}_{\text{U}} = (1.380\times10^{-13}\text{s}^{-1})(1.08\times10^{15}) .$$
This is ${149.04}\text{ Bq}$ or 
$$\boxed{ 4.028\times10^{-9}\text{ Ci} }$$
\end{enumerate}

