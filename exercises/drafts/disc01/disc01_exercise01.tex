%%%%%%%%%%%%%%%%%%%%%%%%%%%%%%%%%% PROBLEM 1 %%%%%%%%%%%%%%%%%%%%%%%%%%%%%%%%%%
\section*{Problem 1}
The radioactive isotope $^{233}$Pa can be produced following neutron capture by $^{232}$Th when the resulting $^{233}$Th decays to $^{233}$Pa. In the neutron flux of a typical reactor, neutron capture in 1 g of $^{232}$Th produces $^{233}$Th at of a rate of $2.0 \times 10^{11}\text{ s}^{-1}$.
\begin{enumerate}[a)]
\item What are the activities (in Ci) of $^{233}$Th and $^{233}$Pa after this sample is irradiated for 1.5 hours?
\item The sample is then placed in storage with no further irradiation so that the $^{233}$Th can decay away. What are
the activities (in Ci) of $^{233}$Th and $^{233}$Pa after 48 hours of storage?
\item The decay of $^{233}$Pa results in $^{233}$U, which is also radioactive. After the above sample has been stored for 1 year what is the $^{233}$U activity in Ci? (Hint: it should not be necessary to set up an additional differential equation to find the $^{233}$U activity.)
\end{enumerate}

\begin{table}[htbp]
	\centering
	\begin{tabular}{|c|c|}
			\hline
			Nucleus		&	Half-life \\
			\hline
			$^{233}$Th	&  $22.3$ min\\
			$^{233}$Pa	&  $27.0$ days\\
			$^{233}$U	&  $1.592 \times 10^5$ yr\\
			\hline
	\end{tabular}
	\label{tab:design-specs}
\end{table}
\begin{center}$1\text{ Ci} = 3.7 \times 10^{10}\text{ s}^{-1}$\end{center}
