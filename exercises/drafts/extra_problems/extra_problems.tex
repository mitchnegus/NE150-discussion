\documentclass{report}
% PACKAGES %
\usepackage[english]{} % Sets the language
\usepackage[margin=2cm]{geometry} % Sets the margin size
\usepackage{graphicx} % Enhanced package for including graphics/figures
\usepackage{float} % Allows figures and tables to be floats
\usepackage{amsmath} % Enhanced math package prepared by the American Mathematical Society
\usepackage{amssymb} % AMS symbols package
\usepackage{breqn} % Allows equation breaking over multiple lines
\usepackage{bm} % Allows you to use \bm{} to make any symbol bold
\usepackage{verbatim} % Allows you to include code snippets
\usepackage{setspace} % Allows you to change the spacing between lines at different points in the document
\usepackage{parskip} % Allows you alter the spacing between paragraphs
\usepackage{multicol} % Allows text division into multiple columns
\usepackage{units} % Allows fractions to be expressed diagonally instead of vertically
\usepackage{booktabs,multirow,multirow} % Gives extra table functionality
\usepackage{enumerate} % Allows you to use customizable bullets
\usepackage{fancyhdr} % Allows customizable headers and footers
\usepackage{tikz} % Allows the creation of diagrams
	\usetikzlibrary{shapes.geometric, arrows}
	\tikzstyle{isotope} = [rectangle, 
					  minimum width=2cm, 
					  minimum height=1.25cm,
					  text centered, 
					  draw=black, 
					 ]
	\tikzstyle{decay} = [rectangle, 
					      rounded corners,
					      minimum width=2cm, 
					      minimum height=1.5cm,
					      text centered, 
					      draw=white, 
					     ]
	\tikzstyle{placeholder} = [rectangle,
					      minimum width=2cm,
					      minimum height=1cm,
					      draw=white,
					      ]
	\tikzstyle{arrow} = [thick,->,>=stealth]
	\tikzstyle{farrow} = [ultra thick,->,>=stealth]
	
% Set path to figure image files
\graphicspath{ {fig/} }

% Set some custom shortcuts
\newcommand{\tab}{\-\hspace{1.5cm}}
\newcommand{\lap}{\nabla^2}
\newcommand{\p}{\partial}

% Set the header on the first page and copyright on all pages
\fancypagestyle{FirstPage}{
\chead{\textbf{Nuclear Engineering 150 -- Discussion Section}}
\rfoot{\small \copyright~2019 Mitchell Negus}
}
\fancypagestyle{EveryPage}{
\rfoot{\small \copyright~2019 Mitchell Negus}
}
\pagestyle{EveryPage}


\begin{document}

\thispagestyle{FirstPage}
\begin{center}
\textbf{\large Extra problems to save for review/backup}
\end{center}
\vspace{1cm}


{\huge More relevant}

\section*{Problem}

A reactor is operating for a long time at some known power density $P_0$. Then, it instantaneously changes power to some power density $P_1$. One fission product of interest is $^{135}$Xe, though it has a neglible yield from the initial fission reaction. $^{135}$Xe precursors $^{135}$Te and $^{135}$I are produced with a combined yield of approximately 6\%, before decaying via $\beta^{-}$ decay to $^{135}$I and $^{135}$Xe respectively. Find the number density of $^{135}$Xe as a function of time after the power change. (Your solution may be left as variables)

\begin{table}[htbp]
	\centering
	\begin{tabular}{|c|c|c|}
			\hline
			Nucleus		&	Half-life 	& Thermal $\sigma_{\text{a}}$ \\
			\hline
			$^{135}$Te	&  $19.0$ s 	& $\sim 0$\\
			$^{135}$I	&  $6.6$ hr 	& $\sim 0$\\
			$^{135}$Xe	&  $9.2$ hr 	& $2.6 \times 10^6$ barns \\
			\hline
	\end{tabular}
	\label{tab:design-specs}
\end{table}


\textbf{Walkthrough of neutron slowing down}\\
(1) Generate a neutron from fission-neutron energy spectrum\\
(2) Find cross section at that point\\
(3) Determine if more likely to scatter/absorb; choose higher prob ((hopefully scatter))\\
(4) Decrease by average energy loss\\
(5) Repeat steps 2/3 until absorbed\\
Identify interaction points on fission cross section plot... show "skipped resonances"

\vspace{4cm}
{\huge Less relevant}

\section*{Problem}

Recall from mechanics that centripetal force is $F_{\text{cent}} = -\frac{mv^2}{r}$ and recall from E\&M that the Coulombic force is $F_{\text{coul}} = -\frac{Ze^2}{r^2}$. Solve for the Bohr radius of the orbit of an electron on hydrogen, assuming the angular momentum $L = mvr$ is quantized multiples of $\hbar$ ($1\hbar, \; 2\hbar,\; 3\hbar$, etc). Compare this to the measured value of $5.2917721067(12)\times10^{11} \text{\AA}$ , the most probable distance between an electron in the ground state and the nucleus of a hydrogen atom.






\end{document}

