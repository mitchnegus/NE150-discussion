%% LaTeX template prepared by Sami Lewis for students in NE 101/210M, fall 2016
%% NOTES:
%%			- This is a simple, bare-bones template for the purposes of typing up homeworks. 
%%				There are many more advanced things you can add if you wish to do so.
%%			- You can make any changes to the template that you'd like.
%%			- Things in all caps are things you need to fill in. Make sure to change them
%%				all, including your name.
%%			-	There are lots of other, nicer free templates online too!
%%			- Run the template once without making any changes so that you can install
%%				all of the packages and see what the formatting looks like.

\documentclass{report}
% PACKAGES %
\usepackage[english]{} % Sets the language
\usepackage[margin=2cm]{geometry} % Sets the margin size
\usepackage{graphicx} % Enhanced package for including graphics/figures
\usepackage{float} % Allows figures and tables to be floats
\usepackage{amsmath} % Enhanced math package prepared by the American Mathematical Society
\usepackage{amssymb} % AMS symbols package
\usepackage{bm} % Allows you to use \bm{} to make any symbol bold
\usepackage{verbatim} % Allows you to include code snippets
\usepackage{setspace} % Allows you to change the spacing between lines at different points in the document
\usepackage{parskip} % Allows you alter the spacing between paragraphs
\usepackage{multicol} % Allows text division into multiple columns
\usepackage{units} % Allows fractions to be expressed diagonally instead of vertically
\usepackage{booktabs,multirow,multirow} % Gives extra table functionality
\usepackage{enumerate}
\newcommand{\tab}{\-\hspace{1.5cm}}

% Set path to figure image files
\graphicspath{ {fig/} }

\begin{document}

\begin{center}
\textbf{\large Nuclear Engineering 150 -- Discussion Section}\\ 
\textbf{Extra problems to save for review/backup}
\end{center}
\vspace{1cm}


{\huge More relevant}

\begin{center}
$<$ None at the moment $>$
\end{center}

\vspace{4cm}
{\huge Less relevant}

\section*{Problem}

Recall from mechanics that centripetal force is $F_{\text{cent}} = -\frac{mv^2}{r}$ and recall from E\&M that the Coulombic force is $F_{\text{coul}} = -\frac{Ze^2}{r^2}$. Solve for the Bohr radius of the orbit of an electron on hydrogen, assuming the angular momentum $L = mvr$ is quantized multiples of $\hbar$ ($1\hbar, \; 2\hbar,\; 3\hbar$, etc). Compare this to the measured value of $5.2917721067(12)\times10^{11} \text{\AA}$ , the most probable distance between an electron in the ground state and the nucleus of a hydrogen atom.






\end{document}

