\section*{Problem 2 Solution}

\begin{enumerate}[a)]

\item 

The total mass-energy of the excited $^{236}$U atom is the sum of the masses of the reactants: the $^{235}$U atom and the neutron.
\begin{align*}
m(^{236}\text{U}^*)	&= m(^{235}\text{U}) + m_n \\
					&= 235.043930\text{ amu} + 1.00866492\text{ amu} \\
					&= 236.052595\text{ amu}
\end{align*}
The excitation energy of $^{236}$U is the difference between the total mass-energy of the system and the rest-mass of $^{236}$U (multiplied by $c^2$, converting the mass-energy in amu to MeV).
\begin{align*}
E_{\text{ex}}	&= \left[m(^{236}\text{U}^*) - m(^{236}\text{U})\right]c^2 \\
				&= \left[236.052595\text{ amu} - 236.045568\text{ amu}\right]c^2 \\
				&= (0.007027\text{ amu})c^2 \\
E_{\text{ex}}	&= 6.545\text{ MeV} 
\end{align*}


\item

The total mass-energy of the excited $^{239}$U atom is the sum of the masses of the reactants: the $^{238}$U atom and the neutron.
\begin{align*}
m(^{239}\text{U}^*)	&= m(^{238}\text{U}) + m_n \\
					&= 238.050788\text{ amu} + 1.00866492\text{ amu} \\
					&= 239.059453\text{ amu}
\end{align*}
The excitation energy of $^{239}$U is the difference between the total mass-energy of the system and the rest-mass of $^{239}$U (multiplied by $c^2$, converting the mass-energy in amu to MeV).
\begin{align*}
E_{\text{ex}}	&= \left[m(^{239}\text{U}^*) - m(^{239}\text{U})\right]c^2 \\
				&= \left[239.059453\text{ amu} - 239.054293\text{ amu}\right]c^2 \\
				&= (0.005160\text{ amu})c^2 \\
E_{\text{ex}}	&= 4.806\text{ MeV} 
\end{align*}


\item

The 6.545 MeV excitation energy of the $^{236}$U is greater than the 6.2 MeV activation energy of the fission process for that nucleus. This means that even when a $^{235}$U nucleus absorbs a neutron with zero kinetic energy, fission is possible---$^{235}$U is fissile. $^{238}$U does not exhibit this property. When $^{238}$U absorbs a neutron and forms $^{239}$U, the excitation energy of 4.806 MeV is less than the activation energy of 6.6 MeV for fission to occur. This means that the absorbed neutron must have more than about 1.8 MeV of kinetic energy to trigger fission, and so $^{238}$U is fissionable.


\item

This absorption reaction is given by
$$ ^{238}\text{U} + n + 2\text{ MeV} = ^{132}\text{Sn} + ^{106}\text{Mo} + n + \blacksquare\text{ MeV} .$$
Using the masses provided, we can calculate the mass-energy of the reactants to be
\begin{align*}
E_r	&= \left[m(^{238}\text{U}) + m_n\right]c^2 \\
	&= \left(238.050788\text{ amu} + 1.00866492\text{ amu}\right)c^2 \\
	&= 222.673\text{ GeV},
\end{align*}
and the mass energy of the products to be
\begin{align*}
E_p	&= \left[m(^{132}\text{Sn}) + m(^{106}\text{Mo}) + m_n\right]c^2 \\
	&= \left(131.917816\text{ amu} + 105.918137\text{ amu} + 1.00866492\text{ amu}\right)c^2 \\
	&= 222.473\text{ GeV}
\end{align*}
When we add the 2 MeV of kinetic energy, $T$ of the incoming neutron, we find that the $Q$-value of this fission reaction is
$$ E_r + T - E_p = 222.673\text{ GeV}
 + 0.002\text{ GeV} - 222.473\text{ GeV} = 205\text{ MeV} $$
We are told that the product neutron carries 2.5\% of this energy,
\begin{align*}
E_n	&= 0.025(205\text{ MeV}) \\
	&= 5.125\text{ MeV}. 
\end{align*}

\underline{This is more than the 1.8 MeV of kinetic energy required to trigger another fission in $^{238}$U, and fission may occur.}
\end{enumerate}
\-\\
{\small *Remember that not all neutrons are born with this energy, but rather in an energy spectrum. In reality, many neutrons produced from a $^{238}$U fission event will not cause subsequent fission events.} 

