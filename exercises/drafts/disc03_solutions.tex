\documentclass{report}
% PACKAGES %
\usepackage[english]{} % Sets the language
\usepackage[margin=2cm]{geometry} % Sets the margin size
\usepackage{graphicx} % Enhanced package for including graphics/figures
\usepackage{float} % Allows figures and tables to be floats
\usepackage{amsmath} % Enhanced math package prepared by the American Mathematical Society
\usepackage{amssymb} % AMS symbols package
\usepackage{bm} % Allows you to use \bm{} to make any symbol bold
\usepackage{verbatim} % Allows you to include code snippets
\usepackage{setspace} % Allows you to change the spacing between lines at different points in the document
\usepackage{parskip} % Allows you alter the spacing between paragraphs
\usepackage{multicol} % Allows text division into multiple columns
\usepackage{units} % Allows fractions to be expressed diagonally instead of vertically
\usepackage{booktabs,multirow,multirow} % Gives extra table functionality
\usepackage{enumerate}
\newcommand{\tab}{\-\hspace{1.5cm}}

% Set path to figure image files
\graphicspath{ {fig/} }

% Use if statement to hide problem solutions
\newif\ifeqns
\eqnsfalse

\begin{document}

\begin{center}
\textbf{\large Nuclear Engineering 150 -- Discussion Section}\\ 
\textbf{Team Exercise Solutions \#3}
\end{center}

%%%%%%%%%%%%%%%%%%%%%%%%%%%%%%%%%% PROBLEM 1 %%%%%%%%%%%%%%%%%%%%%%%%%%%%%%%%%%
\section*{Problem 1}

A reactor is operating for a long time at some known power density $P_0$. Then, it instantaneously changes power to some power density $P_1$. One fission product of interest is $^{135}$Xe, though it has a neglible yield from the initial fission reaction. $^{135}$Xe precursors $^{135}$Te and $^{135}$I are produced with a combined yield of approximately 6\%, before decaying via $\beta^{-}$ decay to $^{135}$I and $^{135}$Xe respectively. Find the number density of $^{135}$Xe as a function of time after the power change. (Your solution may be left as variables)

\begin{table}[htbp]
	\centering
	\begin{tabular}{|c|c|c|}
			\hline
			Nucleus		&	Half-life 	& Thermal $\sigma_{\text{a}}$ \\
			\hline
			$^{135}$Te	&  $19.0$ s 	& $\sim 0$\\
			$^{135}$I	&  $6.6$ hr 	& $\sim 0$\\
			$^{135}$Xe	&  $9.2$ hr 	& $2.6 \times 10^6$ barns \\
			\hline
	\end{tabular}
	\label{tab:design-specs}
\end{table}




\section*{Problem 1 Solution}

We will create the following simple decay chain graphic, built from the information provided in the problem, to visualize the processes described in the problem.

GRAPHIC GRAPHIC GRAPHIC GRAPHIC GRAPHIC GRAPHIC GRAPHIC GRAPHIC GRAPHIC GRAPHIC \\
GRAPHIC GRAPHIC GRAPHIC GRAPHIC GRAPHIC GRAPHIC GRAPHIC GRAPHIC GRAPHIC GRAPHIC \\
GRAPHIC GRAPHIC GRAPHIC GRAPHIC GRAPHIC GRAPHIC GRAPHIC GRAPHIC GRAPHIC GRAPHIC \\
GRAPHIC GRAPHIC GRAPHIC GRAPHIC GRAPHIC GRAPHIC GRAPHIC GRAPHIC GRAPHIC GRAPHIC \\

Starting off, since this is a problem related to decay, we will start from the usual equation for changes in a quantity of radionuclides.

$$ \frac{dN}{dt} = \text{production} - \text{losses} $$

First, we find the neutron production. We are told that after the transition, the reactor is now generating with power density $P_1$. 

The amount of $^{135}$Xe is dependent on its parents, $^{135}$Te and $^{135}$I. Since the half-life of $^{135}$Te (19.0 s) is practically insignificant in comparison to the multi-hour half-lives of its daughters (to be exact, $T_{\nicefrac{1}{2},\text{Te135}} = 0.08\,T_{\nicefrac{1}{2},\text{I135}}$ and $T_{\nicefrac{1}{2},\text{Te135}} = 0.06\,T_{\nicefrac{1}{2},\text{Xe135}}$), we can treat it as instantaneously decaying into iodine.

Now, we can recognize ...



\newpage
%%%%%%%%%%%%%%%%%%%%%%%%%%%%%%%%%% PROBLEM 2 %%%%%%%%%%%%%%%%%%%%%%%%%%%%%%%%%%
\section*{Problem 2}

\begin{enumerate}[a)]
\item Find the excitation energy in $^{236}$U when a neutron with zero kinetic energy is absorbed by $^{235}$U. 
\item Find the excitation energy in $^{239}$U when a neutron with zero kinetic energy is absorbed by $^{238}$U. 
\item The activation energy for $^{236}$U is 6.2 MeV and the activation energy for $^{239}$U is 6.6 MeV. Will fission occur in each of these cases? Identify ${235}$U and ${238}$U as fissile or fissionable and explain.
\item A $^{238}$U nuclei absorbs a 2 MeV neutron and fissions into $^{132}$Sn, $^{106}$Mo, and a neutron. If the neutron is produced with 2.5\% of the total energy released in the reaction, does it have enough energy to fission another $^{238}$U atom?
\end{enumerate}

\begin{table}[htbp]
	\centering
	\begin{tabular}{|c|c|}
			\hline
			Nucleus		&	\tab\- Mass \tab\- \\
			\hline	
			$n$			&  1.00866492 amu \\
			$^{106}$Mo	&  105.918137 amu \\
			$^{132}$Sn	&  131.917816 amu \\
			$^{235}$U	&  235.043930 amu \\
			$^{236}$U	&  236.045568 amu \\	
			$^{238}$U	&  238.050788 amu \\
			$^{239}$U	&  239.054293 amu \\
			\hline
	\end{tabular}
	\label{tab:design-specs}
\end{table}



\section*{Problem 2 Solution}

\subsection*{a.)}

The total mass-energy of the excited $^{236}$U atom is the sum of the masses of the reactants: the $^{235}$U atom and the neutron.
\begin{align*}
m(^{236}\text{U}^*)	&= m(^{235}\text{U}) + m_n \\
					&= 235.043930\text{ amu} + 1.00866492\text{ amu} \\
					&= 236.052595\text{ amu}
\end{align*}
The excitation energy of $^{236}$U is the difference between the total mass-energy of the system and the rest-mass of $^{236}$U (multiplied by $c^2$, converting the mass-energy in amu to MeV).
\begin{align*}
E_{\text{ex}}	&= \left[m(^{236}\text{U}^*) - m(^{236}\text{U})\right]c^2 \\
				&= \left[236.052595\text{ amu} - 236.045568\text{ amu}\right]c^2 \\
				&= (0.007027\text{ amu})c^2 \\
E_{\text{ex}}	&= 6.545\text{ MeV} 
\end{align*}


\subsection*{b.)}

The total mass-energy of the excited $^{239}$U atom is the sum of the masses of the reactants: the $^{238}$U atom and the neutron.
\begin{align*}
m(^{239}\text{U}^*)	&= m(^{238}\text{U}) + m_n \\
					&= 238.050788\text{ amu} + 1.00866492\text{ amu} \\
					&= 239.059453\text{ amu}
\end{align*}
The excitation energy of $^{239}$U is the difference between the total mass-energy of the system and the rest-mass of $^{239}$U (multiplied by $c^2$, converting the mass-energy in amu to MeV).
\begin{align*}
E_{\text{ex}}	&= \left[m(^{239}\text{U}^*) - m(^{239}\text{U})\right]c^2 \\
				&= \left[239.059453\text{ amu} - 239.054293\text{ amu}\right]c^2 \\
				&= (0.005160\text{ amu})c^2 \\
E_{\text{ex}}	&= 4.806\text{ MeV} 
\end{align*}


\subsection*{c.)}

The 6.545 MeV excitation energy of the $^{236}$U is greater than the 6.2 MeV activation energy of the fission process for that nucleus. This means that even when a $^{235}$U nucleus absorbs a neutron with zero kinetic energy, fission is possible---$^{235}$U is fissile. $^{238}$U does not exhibit this property. When $^{238}$U absorbs a neutron and forms $^{239}$U, the excitation energy of 4.806 MeV is less than the activation energy of 6.6 MeV for fission to occur. This means that the absorbed neutron must have more than about 1.8 MeV of kinetic energy to trigger fission, and so $^{238}$U is fissionable.

This process is given by the equation...

\subsection*{d.)}




\end{document}

