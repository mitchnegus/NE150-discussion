\documentclass{report}
% PACKAGES %
\usepackage[english]{} % Sets the language
\usepackage[margin=2cm]{geometry} % Sets the margin size
\usepackage{graphicx} % Enhanced package for including graphics/figures
\usepackage{float} % Allows figures and tables to be floats
\usepackage{amsmath} % Enhanced math package prepared by the American Mathematical Society
\usepackage{amssymb} % AMS symbols package
\usepackage{bm} % Allows you to use \bm{} to make any symbol bold
\usepackage{verbatim} % Allows you to include code snippets
\usepackage{setspace} % Allows you to change the spacing between lines at different points in the document
\usepackage{parskip} % Allows you alter the spacing between paragraphs
\usepackage{multicol} % Allows text division into multiple columns
\usepackage{units} % Allows fractions to be expressed diagonally instead of vertically
\usepackage{booktabs,multirow,multirow} % Gives extra table functionality
\usepackage{enumerate}
\newcommand{\tab}{\-\hspace{1.5cm}}

% Set path to figure image files
\graphicspath{ {fig/} }

\usepackage{tikz} % Allows the creation of diagrams
	\usetikzlibrary{shapes.geometric, arrows}
	\tikzstyle{isotope} = [rectangle, 
					  minimum width=2cm, 
					  minimum height=1.25cm,
					  text centered, 
					  draw=black, 
					 ]
	\tikzstyle{decay} = [rectangle, 
					      rounded corners,
					      minimum width=2cm, 
					      minimum height=1.5cm,
					      text centered, 
					      draw=white, 
					     ]
	\tikzstyle{placeholder} = [rectangle,
					      minimum width=2cm,
					      minimum height=1cm,
					      draw=white,
					      ]
	\tikzstyle{arrow} = [thick,->,>=stealth]
	\tikzstyle{farrow} = [ultra thick,->,>=stealth]

% Use if statement to hide problem solutions
\newif\ifeqns
\eqnsfalse

\begin{document}

\begin{center}
\textbf{\large Nuclear Engineering 150 -- Discussion Section}\\ 
\textbf{Team Exercise Solutions \#3}
\end{center}

%%%%%%%%%%%%%%%%%%%%%%%%%%%%%%%%%% PROBLEM 1 %%%%%%%%%%%%%%%%%%%%%%%%%%%%%%%%%%
\section*{Problem 1}

A reactor is operating for a long time at some known power density $P_0$. Then, it instantaneously changes power to some power density $P_1$. One fission product of interest is $^{135}$Xe, though it has a neglible yield from the initial fission reaction. $^{135}$Xe precursors $^{135}$Te and $^{135}$I are produced with a combined yield, $y$, of approximately 6\%, before decaying via $\beta^{-}$ decay to $^{135}$I and $^{135}$Xe respectively. Find the number density of $^{135}$Xe as a function of time after the power change. (For convenience, let $Q_f$ be the energy produced in a fission reaction, and let $\phi$ be the flux in the reactor. Your answer may be left in terms of these variables.)

\begin{table}[htbp]
	\centering
	\begin{tabular}{|c|c|c|}
			\hline
			Nucleus		&	Half-life 	& Thermal $\sigma_{\text{a}}$ \\
			\hline
			$^{135}$Te	&  $19.0$ s 	& $\sim 0$\\
			$^{135}$I	&  $6.6$ hr 	& $\sim 0$\\
			$^{135}$Xe	&  $9.2$ hr 	& $2.6 \times 10^6$ barns \\
			\hline
	\end{tabular}
	\label{tab:design-specs}
\end{table}



\section*{Problem 1 Solution}

We will create the following simple decay chain graphic, built from the information provided in the problem, to visualize the processes described in the problem.

\begin{center}
\begin{tikzpicture}[node distance=3.25cm]
\node (blank) [placeholder]{};
\node (fission) [isotope, align=center, right of =blank, xshift=2cm, yshift=0.75cm] {\large \textbf{Fission}};
\node (tellurium135) [isotope, align=center, left of =blank] {\large$^{135}$Te};
\node (iodine135) [isotope, align=center, below of =tellurium135, right of =tellurium135] {\large$^{135}$I};
\node (xenon135) [isotope, align=center, below of =iodine135, right of =iodine135] {\large$^{135}$Xe};
\node (xenon136) [placeholder, align=center, left of =xenon135, xshift=-0.5cm] {};
\node (cesium135) [placeholder, align=center, below of =xenon135, right of =xenon135, xshift=-0.9cm, yshift=1cm]{};
\node (yield) [decay, align=center, right of =tellurium135, xshift=3.2cm, yshift=0cm,]{\large \textbf{6\%}};
\node (beta-te-i) [decay, align=center, right of =tellurium135, xshift=-0.5cm, yshift=-1.2cm] {\large$\beta^-$ \\ \footnotesize $t_{\nicefrac{1}{2}} = 19.0$ s};
\node (beta-i-xe) [decay, align=center, right of =iodine135, xshift=-0.5cm, yshift=-1.2cm] {\large$\beta^-$ \\ \footnotesize $t_{\nicefrac{1}{2}} = 6.6$ h};
\node (beta-xe-cs) [decay, align=center, right of =xenon135, xshift=-0.7cm, yshift=-1cm] {\large$\beta^-$ \\ \footnotesize $t_{\nicefrac{1}{2}} = 9.2$ h};
\node (abs-xe-cs) [decay, align=center, below of =xenon136, xshift=1.6cm, yshift=2.7cm] {\large$\sigma_a \phi$};

\draw [farrow] (fission) -- (tellurium135);
\draw [farrow] (fission) -- (iodine135);
\draw [arrow] (tellurium135) -- (iodine135);
\draw [arrow] (iodine135) -- (xenon135);
\draw [arrow] (xenon135) -- (cesium135);
\draw [arrow] (xenon135) -- (xenon136);

\end{tikzpicture}
\end{center}

Starting off, since this is a problem related to decay, we will start from the usual equation for changes in the number density of radionuclides. (This equation may be more familiar in terms of absolute quantity of a material,$N$, in atoms; $n$ is the number density, found by dividing $N$ by the volume containing the $N$ nuclei)
\begin{equation}
\label{general}
\frac{dn}{dt} = \text{production} - \text{loss from decay}
\end{equation}

Nuclide production may occur directly from the fission reaction or from the decay of other fission products into that nuclide. Nuclide losses will be caused by the decay of the radionuclide, or when that nucleus absorbs neutrons in the reactor's high-neutron-flux environment.

For $^{135}$Xe, the fission product of interest, we can decompose this differential equation more explicitly as

\begin{equation}
\label{text-dn-xenon}
\frac{dn_{\text{Xe}}}{dt} = \text{production from }^{135}\text{I} - \text{loss from decay of }^{135}\text{Xe} - \text{loss from absorption} .
\end{equation}

In this equation, the production due to the decay of iodine is just the activity (per volume) of iodine,
$$ \mathcal{A}_{\text{I}} = \lambda_{\text{I}}n_{\text{I}}, $$
and the loss due to the decay of xenon is just the activity (per volume) of xenon,
$$ \mathcal{A}_{\text{Xe}} = \lambda_{\text{Xe}}n_{\text{Xe}}. $$
The loss due to absorption can be calculated by knowing that the absorption rate of neutrons on $^{135}$Xe is $R = \Sigma_a \phi$, where $\phi$ is the neutron flux in the reactor. Since $\Sigma_{a,\text{Xe}}$ is defined as $\sigma_{a,\text{Xe}} n_{\text{Xe}}$, the absorptive losses are
$$ R_{\text{Xe,abs}} \sigma_{a,\text{Xe}} n_{\text{Xe}} \phi .$$
The full equation for xenon can be written,
$$ \frac{dn_{\text{Xe}}}{dt} = \mathcal{A}_{\text{I}} - \mathcal{A}_{\text{Xe}} - R_{\text{Xe,abs}} $$
or
$$ \frac{dn_{\text{Xe}}}{dt} = \lambda_{\text{I}}n_{\text{I}} - \lambda_{\text{Xe}}n_{\text{Xe}} - \sigma_{a,\text{Xe}} n_{\text{Xe}} \phi .$$
We factor out $n_{\text{Xe}}$ from the last two terms to get
\begin{equation}
\label{dn-xenon}
\frac{dn_{\text{Xe}}}{dt} = \lambda_{\text{I}}n_{\text{I}} - (\lambda_{\text{Xe}} + \sigma_a\phi) n_{\text{Xe}}
\end{equation}
In this form, it is easier to see that the number densities are the only two unknown functions in this differential equation. If we could express $n_{\text{I}}$ in terms of $t$, we could solve the equation for $n_{\text{Xe}}$.

To express $n_{\text{I}}$ in terms of $t$, we again use equation (\ref{general}). Proceeding in a similar fashion as we did for xenon, we explicitly write out the equation for iodine akin to equation (\ref{text-dn-xenon})
\begin{equation}
\label{text-dn-iodine}
\frac{dn_{\text{I}}}{dt} = \text{production from fission} + \text{production from }^{135}\text{Te} - \text{loss from decay of }^{135}\text{Xe}.
\end{equation}
We have ignored absorptive losses since the thermal absorption cross section for iodine is essentially zero. 

The production of iodine due to fission can be calculated from the iodine-yield of the fission reaction. While this information is not provided, the combined yield, $y$, of iodine and tellurium \textit{is} given. Since the half-life of $^{135}$Te (19.0 s) is practically insignificant in comparison to the multi-hour half-lives of its daughters (to be exact, $t_{\nicefrac{1}{2},\text{Te135}} = 0.0008(t_{\nicefrac{1}{2},\text{I135}})$ and $t_{\nicefrac{1}{2},\text{Te135}} = 0.0006(t_{\nicefrac{1}{2},\text{Xe135}})$), we can treat it as instantaneously decaying into iodine. With this assumption, the production of iodine due to fission is just the fission rate density multiplied by the combined yield of iodine and tellurium. The fission rate density can be found by dividing the power density by the energy per fission, $Q_f$, so
$$ R_{\text{I,fission}} = y\frac{P}{Q_f} .$$
The loss from the decay of iodine is just the activity of iodine,
$$ \mathcal{A}_{\text{I}} = \lambda_{\text{I}}n_{\text{I}}. $$
The full equation for iodine can be written,
$$ \frac{dn_{\text{I}}}{dt} = R_{\text{I,fission}} + \mathcal{A}_{\text{I}} $$
or 
\begin{equation}
\label{dn-iodine}
\frac{dn_{\text{I}}}{dt} = y\frac{P}{Q_f} + \lambda_{\text{I}}n_{\text{I}} .
\end{equation}
Considering that the power after the power change is $P=P_1$, we can solve for number density of iodine after the change, $n_{\text{I},1}$.
\begin{equation}
\label{n-iodine1}
n_{\text{I},1}(t) = \frac{y P_1}{\lambda_{\text{I}}Q_f}\left(1-e^{-\lambda_{\text{I}}t}\right) + n_{\text{I},1}(0)e^{-\lambda_{\text{I}}t}.
\end{equation}
\begin{center}(for a detailed derivation of this, see the appendix)\end{center}
Looking closely at this equation, we find that the only unknown quantity here is $n_{\text{I},1}(0)$. While using $t=0$ in this equation gives us the trivial solution that $n_{\text{I},1}(0) = n_{\text{I},1}(0)$, we note that equation (\ref{dn-iodine}) could be solved identically for times before the power change by simply replacing $P_1$ by $P_0$. 
\begin{equation}
\label{n-iodine0}
n_{\text{I},0}(t) = \frac{y P_0}{\lambda_{\text{I}}Q_f}\left(1-e^{-\lambda_{\text{I}}t}\right) + n_{\text{I},0}(0)e^{-\lambda_{\text{I}}t}.
\end{equation}
We are told that our reactor has been operating for a long time. If we assume that this time period is sufficiently long that we can approximate the time of the power change as $t=\infty$, then equation (\ref{n-iodine0}) reduces to
$$ n_{\text{I},0}(\infty) = \frac{y P_0}{\lambda_{\text{I}}Q_f} ,$$
and
$$ n_{\text{I},0}(\infty) = n_{\text{I},1}(0) .$$
We replace this value of $n_{\text{I},1}(0)$ in equation (\ref{n-iodine1}) to get
\begin{equation}
\label{n-iodine1-comp}
n_{\text{I},1}(t) = \frac{y P_1}{\lambda_{\text{I}}Q_f}\left(1-e^{-\lambda_{\text{I}}t}\right) + \frac{y P_0}{\lambda_{\text{I}}Q_f}e^{-\lambda_{\text{I}}t}.
\end{equation}
This equation has $n_{\text{I}}$ exclusively as a function of $t$, and so we can now use it back in equation (\ref{dn-xenon}) to find the number density of xenon after the power change. 
\begin{align*}
\frac{dn_{\text{Xe},1}}{dt}	&= \lambda_{\text{I}}\left(\frac{y P_1}{\lambda_{\text{I}}Q_f}\left(1-e^{-\lambda_{\text{I}}t}\right) + \frac{y P_0}{\lambda_{\text{I}}Q_f}e^{-\lambda_{\text{I}}t}\right) - (\lambda_{\text{Xe}} + \sigma_{a,\text{Xe}}\phi) n_{\text{Xe},1} \\
							&= \frac{y P_1}{Q_f}\left(1-e^{-\lambda_{\text{I}}t}\right) + \frac{y P_0}{Q_f}e^{-\lambda_{\text{I}}t} - (\lambda_{\text{Xe}} + \sigma_{a,\text{Xe}}\phi) n_{\text{Xe},1}\\
\end{align*}
In this equation we will introduce a parameter called the "effective destruction constant", $\lambda_{\text{Xe}}^{\text{eff}} = \lambda_{\text{Xe}} + \sigma_{a,\text{Xe}} \phi. $ The differential equation above becomes
$$ \frac{dn_{\text{Xe},1}}{dt} = \frac{y P_1}{Q_f}\left(1-e^{-\lambda_{\text{I}}t}\right) + \frac{y P_0}{Q_f}e^{-\lambda_{\text{I}}t} - \lambda_{\text{Xe}}^{\text{eff}} n_{\text{Xe},1} $$

\textbf{Note:} If you've made it this far, you've got the general gist of the problem. Nice work! The remainder get's a little tricky---you need an integrating factor to solve this differential equation---and though the solution looks a bit intimidating, it can begin to reveal some of the complicated behavior of an operating reactor. This is information that will be covered more comprehensively later in the semester, but which is also useful to introduce now. 

When we use the integrating factor, we find
\begin{align*}
N_{\text{Xe},1}(t)	&= N_{\text{Xe},1}(0)e^{-\lambda_{\text{Xe}}^{\text{eff}}t} + \frac{yP_1}{\lambda_{\text{Xe}}^{\text{eff}}Q_f}\left(1-e^{-\lambda_{\text{Xe}}^{\text{eff}} t}\right) - \frac{yP_1}{(\lambda_{\text{Xe}}^{\text{eff}}-\lambda_{\text{I}}) Q_f} \left(e^{-\lambda_{\text{I}}t} - e^{-\lambda_{\text{Xe}}^{\text{eff}}t}\right) + \frac{yP_0}{(\lambda_{\text{Xe}}^{\text{eff}}-\lambda_{\text{I}}) Q_f} \left(e^{-\lambda_{\text{I}}t} - e^{-\lambda_{\text{Xe}}^{\text{eff}}t}\right) \\
					&= N_{\text{Xe},1}(0)e^{-\lambda_{\text{Xe}}^{\text{eff}}t} + \frac{yP_1}{\lambda_{\text{Xe}}^{\text{eff}}Q_f}\left(1-e^{-\lambda_{\text{Xe}}^{\text{eff}} t}\right) - \left(\frac{yP_1}{(\lambda_{\text{Xe}}^{\text{eff}}-\lambda_{\text{I}}) Q_f} - \frac{yP_0}{(\lambda_{\text{Xe}}^{\text{eff}}-\lambda_{\text{I}}) Q_f}\right) \left(e^{-\lambda_{\text{I}}t} - e^{-\lambda_{\text{Xe}}^{\text{eff}}t}\right) \\
					&= N_{\text{Xe},1}(0)e^{-\lambda_{\text{Xe}}^{\text{eff}}t} + \frac{yP_1}{\lambda_{\text{Xe}}^{\text{eff}}Q_f}\left(1-e^{-\lambda_{\text{Xe}}^{\text{eff}} t}\right) - \frac{y\left(P_1-P_0\right)}{(\lambda_{\text{Xe}}^{\text{eff}}-\lambda_{\text{I}}) Q_f} \left(e^{-\lambda_{\text{I}}t} - e^{-\lambda_{\text{Xe}}^{\text{eff}}t}\right)
\end{align*}
\begin{center}(for a detailed derivation of this, see the appendix)\end{center}

From this equation, we see that if $P_1 > P_0$, then the sign of the third term is opposite the case when $P_0 > P_1$. Using typical values for all the constants, including flux and energy per fission, this sign change suggests that power increases and decreases will have opposite effects on the xenon population. If the power increases ($P_1 > P_0$) then the xenon population will temporarily decrease. If instead the power decreases ($P_1 < P_0$) then the xenon population will increase for a while, before xenon losses compensate for the production due to decaying iodine. This behavior can lead to situations where an operator is unable to restart a reactor immediately after shutting down (power has decreased, fission product concentrations temporarily jump, and restarting the reactor is unsafe). This was a contributing factor at Chernobyl.



\newpage
%%%%%%%%%%%%%%%%%%%%%%%%%%%%%%%%%% PROBLEM 2 %%%%%%%%%%%%%%%%%%%%%%%%%%%%%%%%%%
\section*{Problem 2}

\begin{enumerate}[a)]
\item Find the excitation energy in $^{236}$U when a neutron with zero kinetic energy is absorbed by $^{235}$U. 
\item Find the excitation energy in $^{239}$U when a neutron with zero kinetic energy is absorbed by $^{238}$U. 
\item The activation energy for $^{236}$U is 6.2 MeV and the activation energy for $^{239}$U is 6.6 MeV. Will fission occur in each of these cases? Identify $^{235}$U and $^{238}$U as fissile or fissionable and explain.
\item A $^{238}$U nuclei absorbs a 2 MeV neutron and fissions into $^{132}$Sn, $^{106}$Mo, and a neutron. If the neutron is produced with 2.5\% of the total energy released in the reaction, does it have enough energy to fission another $^{238}$U atom?
\end{enumerate}

\begin{table}[htbp]
	\centering
	\begin{tabular}{|c|c|}
			\hline
			Nucleus		&	\tab\- Mass \tab\- \\
			\hline	
			$n$			&  1.00866492 amu \\
			$^{106}$Mo	&  105.918137 amu \\
			$^{132}$Sn	&  131.917816 amu \\
			$^{235}$U	&  235.043930 amu \\
			$^{236}$U	&  236.045568 amu \\	
			$^{238}$U	&  238.050788 amu \\
			$^{239}$U	&  239.054293 amu \\
			\hline
	\end{tabular}
	\label{tab:design-specs}
\end{table}



\section*{Problem 2 Solution}

\subsection*{a.)}

The total mass-energy of the excited $^{236}$U atom is the sum of the masses of the reactants: the $^{235}$U atom and the neutron.
\begin{align*}
m(^{236}\text{U}^*)	&= m(^{235}\text{U}) + m_n \\
					&= 235.043930\text{ amu} + 1.00866492\text{ amu} \\
					&= 236.052595\text{ amu}
\end{align*}
The excitation energy of $^{236}$U is the difference between the total mass-energy of the system and the rest-mass of $^{236}$U (multiplied by $c^2$, converting the mass-energy in amu to MeV).
\begin{align*}
E_{\text{ex}}	&= \left[m(^{236}\text{U}^*) - m(^{236}\text{U})\right]c^2 \\
				&= \left[236.052595\text{ amu} - 236.045568\text{ amu}\right]c^2 \\
				&= (0.007027\text{ amu})c^2 \\
E_{\text{ex}}	&= 6.545\text{ MeV} 
\end{align*}


\subsection*{b.)}

The total mass-energy of the excited $^{239}$U atom is the sum of the masses of the reactants: the $^{238}$U atom and the neutron.
\begin{align*}
m(^{239}\text{U}^*)	&= m(^{238}\text{U}) + m_n \\
					&= 238.050788\text{ amu} + 1.00866492\text{ amu} \\
					&= 239.059453\text{ amu}
\end{align*}
The excitation energy of $^{239}$U is the difference between the total mass-energy of the system and the rest-mass of $^{239}$U (multiplied by $c^2$, converting the mass-energy in amu to MeV).
\begin{align*}
E_{\text{ex}}	&= \left[m(^{239}\text{U}^*) - m(^{239}\text{U})\right]c^2 \\
				&= \left[239.059453\text{ amu} - 239.054293\text{ amu}\right]c^2 \\
				&= (0.005160\text{ amu})c^2 \\
E_{\text{ex}}	&= 4.806\text{ MeV} 
\end{align*}


\subsection*{c.)}

The 6.545 MeV excitation energy of the $^{236}$U is greater than the 6.2 MeV activation energy of the fission process for that nucleus. This means that even when a $^{235}$U nucleus absorbs a neutron with zero kinetic energy, fission is possible---$^{235}$U is fissile. $^{238}$U does not exhibit this property. When $^{238}$U absorbs a neutron and forms $^{239}$U, the excitation energy of 4.806 MeV is less than the activation energy of 6.6 MeV for fission to occur. This means that the absorbed neutron must have more than about 1.8 MeV of kinetic energy to trigger fission, and so $^{238}$U is fissionable.


\subsection*{d.)}

This absorption reaction is given by
$$ ^{238}\text{U} + n + 2\text{ MeV} = ^{132}\text{Sn} + ^{106}\text{Mo} + n + \blacksquare\text{ MeV} .$$
Using the masses provided, we can calculate the mass-energy of the reactants to be
\begin{align*}
E_r	&= \left[m(^{238}\text{U}) + m_n\right]c^2 \\
	&= \left(238.050788\text{ amu} + 1.00866492\text{ amu}\right)c^2 \\
	&= 222.673\text{ GeV},
\end{align*}
and the mass energy of the products to be
\begin{align*}
E_p	&= \left[m(^{132}\text{Sn}) + m(^{106}\text{Mo}) + m_n\right]c^2 \\
	&= \left(131.917816\text{ amu} + 105.918137\text{ amu} + 1.00866492\text{ amu}\right)c^2 \\
	&= 222.473\text{ GeV}
\end{align*}
When we add the 2 MeV of kinetic energy, $T$ of the incoming neutron, we find that the $Q$-value of this fission reaction is
$$ E_r + T - E_p = 222.673\text{ GeV}
 + 0.002\text{ GeV} - 222.473\text{ GeV} = 205\text{ MeV} $$
We are told that the product neutron carries 2.5\% of this energy,
\begin{align*}
E_n	&= 0.025(205\text{ MeV}) \\
	&= 5.125\text{ MeV}. 
\end{align*}

\underline{This is more than the 1.8 MeV of kinetic energy required to trigger another fission in $^{238}$U, and fission may occur.}

\-\\
{\small *Remember that not all neutrons are born with this energy, but rather in an energy spectrum. In reality, many neutrons produced from a $^{238}$U fission event will not cause subsequent fission events.} 



\newpage
%%%%%%%%%%%%%%%%%%%%%%%%%%%%%%%%%% APPENDIX %%%%%%%%%%%%%%%%%%%%%%%%%%%%%%%%%%
\section*{Appendix}

\textit{Sorry, I haven't typed up the detailed solutions for these equations just yet. Check back later! In the meantime, you should be able to find this derivation in a detailed discussion of the Bateman equations or, more generally, in a discussion of differential equations.}

\end{document}

