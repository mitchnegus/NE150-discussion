\documentclass{report}
% PACKAGES %
\usepackage[english]{} % Sets the language
\usepackage[margin=2cm]{geometry} % Sets the margin size
\usepackage{graphicx} % Enhanced package for including graphics/figures
\usepackage{float} % Allows figures and tables to be floats
\usepackage{amsmath} % Enhanced math package prepared by the American Mathematical Society
\usepackage{amssymb} % AMS symbols package
\usepackage{bm} % Allows you to use \bm{} to make any symbol bold
\usepackage{verbatim} % Allows you to include code snippets
\usepackage{setspace} % Allows you to change the spacing between lines at different points in the document
\usepackage{parskip} % Allows you alter the spacing between paragraphs
\usepackage{multicol} % Allows text division into multiple columns
\usepackage{units} % Allows fractions to be expressed diagonally instead of vertically
\usepackage{booktabs,multirow,multirow} % Gives extra table functionality
\usepackage{enumerate}
\newcommand{\tab}{\-\hspace{1.5cm}}

% Set path to figure image files
\graphicspath{ {fig/} }

% Use if statement to hide problem solutions
\newif\ifeqns
\eqnsfalse

\begin{document}

\begin{center}
\textbf{\large Nuclear Engineering 150 -- Discussion Section}\\ 
\textbf{Team Exercise Solutions \#3}
\end{center}

%%%%%%%%%%%%%%%%%%%%%%%%%%%%%%%%%% PROBLEM 1 %%%%%%%%%%%%%%%%%%%%%%%%%%%%%%%%%%
\section*{Problem 1}

<<<<<<< HEAD
\textit{Text of problem 1}
=======
A reactor is operating for a long time at some known power density $P_0$. Then, it instantaneously changes power to some power density $P_1$. One fission product of interest is $^{135}$Xe, though it has a neglible yield from the initial fission reaction. $^{135}$Xe precursors $^{135}$Te and $^{135}$I are produced with a combined yield of approximately 6\%, before decaying via $\beta^{-}$ decay to $^{135}$I and $^{135}$Xe respectively. Find the number density of $^{135}$Xe as a function of time after the power change. Let $Q_f$ be the energy produced per fission, approximately $200$ MeV.

\begin{table}[htbp]
	\centering
	\begin{tabular}{|c|c|c|}
			\hline
			Nucleus		&	Half-life 	& Thermal $\sigma_{\text{a}}$ \\
			\hline
			$^{135}$Te	&  $19.0$ s 	& $\sim 0$\\
			$^{135}$I	&  $6.6$ hr 	& $\sim 0$\\
			$^{135}$Xe	&  $9.2$ hr 	& $2.6 \times 10^6$ barns \\
			\hline
	\end{tabular}
	\label{tab:design-specs}
\end{table}
>>>>>>> exercises



\section*{Problem 1 Solution}

<<<<<<< HEAD
=======
We will create the following simple decay chain graphic, built from the information provided in the problem, to visualize the processes described in the problem.

GRAPHIC GRAPHIC GRAPHIC GRAPHIC GRAPHIC GRAPHIC GRAPHIC GRAPHIC GRAPHIC GRAPHIC \\
GRAPHIC GRAPHIC GRAPHIC GRAPHIC GRAPHIC GRAPHIC GRAPHIC GRAPHIC GRAPHIC GRAPHIC \\
GRAPHIC GRAPHIC GRAPHIC GRAPHIC GRAPHIC GRAPHIC GRAPHIC GRAPHIC GRAPHIC GRAPHIC \\
GRAPHIC GRAPHIC GRAPHIC GRAPHIC GRAPHIC GRAPHIC GRAPHIC GRAPHIC GRAPHIC GRAPHIC \\

Starting off, since this is a problem related to decay, we will start from the usual equation for changes in a quantity of radionuclides.

$$ \frac{dN}{dt} = \text{production} - \text{losses} $$

First, we find the neutron production. We are told that after the transition, the reactor is now generating with power density $P_1$. 

The amount of $^{135}$Xe is dependent on its parents, $^{135}$Te and $^{135}$I. Since the half-life of $^{135}$Te (19.0 s) is practically insignificant in comparison to the multi-hour half-lives of its daughters (to be exact, $T_{\nicefrac{1}{2},\text{Te135}} = 0.08\,T_{\nicefrac{1}{2},\text{I135}}$ and $T_{\nicefrac{1}{2},\text{Te135}} = 0.06\,T_{\nicefrac{1}{2},\text{Xe135}}$), we can treat it as instantaneously decaying into iodine.

Now, we can recognize ...

>>>>>>> exercises


\newpage
%%%%%%%%%%%%%%%%%%%%%%%%%%%%%%%%%% PROBLEM 2 %%%%%%%%%%%%%%%%%%%%%%%%%%%%%%%%%%
\section*{Problem 2}

\textit{Text of problem 2}



\section*{Problem 2 Solution}






\end{document}

