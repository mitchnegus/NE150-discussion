\documentclass{report}
% PACKAGES %
\usepackage[english]{} % Sets the language
\usepackage[margin=2cm]{geometry} % Sets the margin size
\usepackage{graphicx} % Enhanced package for including graphics/figures
\usepackage{float} % Allows figures and tables to be floats
\usepackage{amsmath} % Enhanced math package prepared by the American Mathematical Society
\usepackage{amssymb} % AMS symbols package
\usepackage{breqn} % Allows equation breaking over multiple lines
\usepackage{bm} % Allows you to use \bm{} to make any symbol bold
\usepackage{verbatim} % Allows you to include code snippets
\usepackage{setspace} % Allows you to change the spacing between lines at different points in the document
\usepackage{parskip} % Allows you alter the spacing between paragraphs
\usepackage{multicol} % Allows text division into multiple columns
\usepackage{units} % Allows fractions to be expressed diagonally instead of vertically
\usepackage{booktabs,multirow,multirow} % Gives extra table functionality
\usepackage{enumerate} % Allows you to use customizable bullets
\usepackage{fancyhdr} % Allows customizable headers and footers
\usepackage{tikz} % Allows the creation of diagrams
	\usetikzlibrary{shapes.geometric, arrows}
	\tikzstyle{isotope} = [rectangle, 
					  minimum width=2cm, 
					  minimum height=1.25cm,
					  text centered, 
					  draw=black, 
					 ]
	\tikzstyle{decay} = [rectangle, 
					      rounded corners,
					      minimum width=2cm, 
					      minimum height=1.5cm,
					      text centered, 
					      draw=white, 
					     ]
	\tikzstyle{placeholder} = [rectangle,
					      minimum width=2cm,
					      minimum height=1cm,
					      draw=white,
					      ]
	\tikzstyle{arrow} = [thick,->,>=stealth]
	\tikzstyle{farrow} = [ultra thick,->,>=stealth]
	
% Set path to figure image files
\graphicspath{ {fig/} }

% Set some custom shortcuts
\newcommand{\tab}{\-\hspace{1.5cm}}
\newcommand{\lap}{\nabla^2}
\newcommand{\p}{\partial}

% Set the header on the first page and copyright on all pages
\fancypagestyle{FirstPage}{
\chead{\textbf{Nuclear Engineering 150 -- Discussion Section}}
\rfoot{\small \copyright~2019 Mitchell Negus}
}
\fancypagestyle{EveryPage}{
\rfoot{\small \copyright~2019 Mitchell Negus}
}
\pagestyle{EveryPage}


% Use if statement to hide problem 1 solution
\newif\ifeqns
\eqnsfalse

\begin{document}

\begin{center}
\textbf{\large Nuclear Engineering 150 -- Discussion Section}\\ 
\textbf{Team Exercise Solutions \#2}
\end{center}

%%%%% PROBLEM 1 %%%%%
%%%%%%%%%%%%%%%%%%%%%%%%%%%%%%%%%% PROBLEM 1 %%%%%%%%%%%%%%%%%%%%%%%%%%%%%%%%%%
\section*{Problem 1}

Compute the atomic densities of $^{235}$U, $^{238}$U, and O in UO$_2$ when its density is 10.41 g/cm$^3$ and the uranium is enriched to 5 wt\% in $^{235}$U.



\section*{Problem 1 Solution}


\ifeqns
First, let's note that we will use the shorthand U5 and U8 in subscripts to denote $^{235}$U and $^{238}$U respectively. 

To begin we note that the densities of $^{235}$U and $^{238}$U can be found as fractions of the total UO$_2$. Since the partial densities are dependent on the mass of the components, the fractions are equivalent to the weight \% enrichment.
$$ \rho_{\text{UO}_2} = \rho_{\text{U5O}_2} + \rho_{\text{U8O}_2}$$
$$ \begin{array}{lr}
        \rho_{\text{U5O}_2} = w\rho_{\text{UO}_2}    \\
        \rho_{\text{U8O}_2} = (1-w)\rho_{\text{UO}_2}\\
        \end{array} \text{ where } w = 0.05 $$
        
Now, we can use the fact that the number density of a material, $n$, is equal to the material's density, $\rho$, multiplied by Avogadro's number, $N_A$, and divided by the molar mass, $m$, of the material. 
$$ n = \frac{\rho N_A}{m} $$
We find
$$ n_{\text{U5O}_2} = \frac{\rho_{\text{U5O}_2} N_A}{m_{\text{U5O}_2}} $$
$$ n_{\text{U8O}_2} = \frac{\rho_{\text{U8O}_2} N_A}{m_{\text{U8O}_2}} $$
If we make the simplification that $n_{\text{U5}} = n_{\text{U5O}_2}$ and $n_{\text{U8}} = n_{\text{U8O}_2}$, and also substitute our equations for $\rho_{\text{U5O}_2}$ and $\rho_{\text{U8O}_2}$ in terms of $\rho_{\text{UO}_2}$ from above, then we find 
\begin{align*}
n_{\text{U5}} &= \frac{w\rho_{\text{UO}_2} N_A}{m_{\text{U5O}_2}} \\
n_{\text{U8}} &= \frac{(1-w)\rho_{\text{UO}_2} N_A}{m_{\text{U8O}_2}} \\
\end{align*}
At this point, we can also separate our molar masses into molar masses of the components, for which we can find data. 
$$ m_{\text{U5O}_2} = m_{\text{U5}} + 2m_{\text{O}} $$
$$ m_{\text{U8O}_2} = m_{\text{U8}} + 2m_{\text{O}} $$
We now have enough information to solve for both of these quantities, and so we plug in values. We also note that $n_O = 2(n_{\text{U8}} + n_{\text{U8}})$ since each uranium atom is bonded to 2 oxygen atoms.
\begin{align*}
n_{\text{U5}}	&= \frac{w\rho_{\text{UO}_2} N_A}{m_{\text{U5}} + 2m_{\text{O}}}		& \boxed{n_{\text{U5}}= ... }\\
n_{\text{U8}}	&= \frac{(1-w)\rho_{\text{UO}_2} N_A}{m_{\text{U8}} + 2m_{\text{O}}} 	& \boxed{n_{\text{U8}}= ... }\\
n_{\text{O}}	&= 2(n_{\text{U5}} + n_{\text{U8}})  									& \boxed{n_{\text{O}}= ...  }\\
\end{align*}
\else
\textit{This problem is currently on Homework 1 (Spring 2018). The solution will be posted after the homework due date.}
\fi



\newpage
%%%%% PROBLEM 2 %%%%%
%%%%%%%%%%%%%%%%%%%%%%%%%%%%%%%%%% PROBLEM 2 %%%%%%%%%%%%%%%%%%%%%%%%%%%%%%%%%%
\section*{Problem 2}

Free neutrons undergo $\beta^{-}$ decay with a half-life of 10.4 minutes. 
Determine the probability that a neutron will decay before being absorbed in an infinite absorbing material (assume no scattering). 
Estimate this probability for a thermal neutron ($v = 2200$ m/s) in water.



\section*{Problem 2 Solution}


Since we are assuming no scattering, at any given time $t$ there are three possible ``fates'' of our neutron in the next infinitesimal timestep, $dt$. These are (1) the particle continues with no decay or interaction, (2) the neutron decays, or (3) the neutron is absorbed. We want to find the probability that, given a destructive event has taken place (situation 2 or 3), the event was decay, rather than absorption. Formally, we write this as 
$$ \text{p}(E=d | E \in \{d,a\}) $$

which we read as ``the instantaneous probability that an event is decay, given the event was decay or absorption.'' Statistically this is equivalent to
\begin{equation}
\label{prob}
\frac{\text{p}(E=d)}{\text{p}(E \in \{d,a\}} \text{ or } \frac{\text{p}(d)}{\text{p}(d) + \text{p}(a)}.
\end{equation}
(If you're not comfortable with the statistics here, you can reason through this intuitively. We have some event occuring, being either a decay or an absorption collision. The probability of that event depends on the individual probabilities of the independent events, decay and absorption. The probability of that event being a decay is just the fraction of the event's total probability that is attributable to the decay.)

From here on, we will write the argument of the probality function as a subscript, which will allow us to note the probability's dependence on either time or space. (For example, $\text{p}(d) \rightarrow \text{p}_d$.)

Now, we know that the exponential decay law is $N(t) = N_0 e^{-\lambda t}$. The probability that a neutron decays in some time $t$ is then
$$ P_d(t) = 1-\frac{N(t)}{N_0} = 1-e^{-\lambda t} $$
We can calculate that the instantantaneous probability that a neutron decays between times $t$ and $t+dt$ is 
$$ \frac{dP_d(t)}{dt} = \lambda e^{-\lambda t} $$
From this equation, we note that $dP_d(t) = p_d(t)dt$. Remember, $P_d(t)$ is the total probability of decay, and so $p_d(t)$ is the instantaneous probability of decay per time. When multiplied by $dt$, this is $\text{p}_d(t)$.
$$ \text{p}_d(t) = p_d(t) \, dt = \lambda e^{-\lambda t} dt $$
We rewrite this equation in terms of $N(t)$, and  
$$ \text{p}_d(t) = \frac{\lambda N(t)}{N_0} dt$$

Additionally, we know that the attenuation of neutrons moving through a material also follows an exponential function. The number of neutrons remaining after traveling through a material with macroscopic absorption cross section $\Sigma_a$, is
$$ N(x) = N_0 e^{-\Sigma_a x} $$
(note: $\Sigma_a = \sigma_a n$, where $\sigma_a$ is the microscopic absorption cross section and $n$ is the number density of the material in question) 

We repeat the procedure from above, but now in space rather than time. The probability for a collision is
$$ P_a(x) = 1-\frac{N(x)}{N_0} = 1 - e^{-\Sigma_a x} $$
and the instantaneous probability is 
$$ \frac{dP_a(x)}{dt} = \Sigma_a e^{\Sigma_a x} $$
or 
$$ \text{p}_a(x) = p_a(x) \, dx = \Sigma_a e^{\Sigma_a x} dx .$$
Again, eliminating $dx$ and writing in terms of $N(x)$, we have
$$ \text{p}_a(x) = \frac{\Sigma_a N(x)}{N_0} dx $$
Now, we observe that using $\text{p}_d(t)$ and $\text{p}_a(x)$ in equation (\ref{prob}) causes all terms to be dependent on $N$, the number of neutrons. Unfortunately, we have $p_d$ reliant on $N(t)$, the number of neutrons surviving until time $t$, and $\text{p}_a$ reliant on $N(x)$, the number of neutrons reaching distance $x$. We are told, however, that these neutrons are thermal, and so are moving at $v = 2200$m/s. If we say $dx = v \, dt$, then we can rewrite $N(x) dx$ as a function of time! 
$$ N(x) \, dx = N(t) v \, dt $$
and
$$ \text{p}_a(t) = \frac{\Sigma_a N(t) v}{N_0} dt $$
Finally, we do substitute this into equation (\ref{prob}), and get
\begin{align*}
\text{p}(E=d | E \in \{d,a\})	&= \frac{\frac{\lambda N(t)}{N_0} dt}{\frac{\lambda N(t)}{N_0} dt + \frac{\Sigma_a N(t) v}{N_0} dt}\\
								&= \frac{\lambda}{\lambda + \Sigma_a v}\\
								&= \left(1 + \frac{\Sigma_a v}{\lambda}\right)^{-1}\\
\end{align*}
We can look up $\Sigma_a$ to find it is approximately $0.022\text{ cm}^{-1}$ (or $2.2\text{ m}^{-1}$, and can calculate that 10.2 minutes is 612 seconds.  Using this, along with $v = 2200$ m/s, and $\lambda = \frac{\ln 2}{612} = 1.13\times10^{-3}\text{ s}^{-1}$, we can find the probability.
$$ \text{p}(E=d | E \in \{d,a\}) = \left(1 + \frac{(2.2\text{ m}^{-1})(2200\text{ m/s})}{1.13\times10^{-3}\text{ s}^{-1}}\right)^{-1} $$
$$\boxed{ \text{p}(E=d | E \in \{d,a\}) = 2.33\times10^{-7} }$$



\newpage
%%%%% PROBLEM 3 %%%%%
%%%%%%%%%%%%%%%%%%%%%%%%%%%%%%%%%% PROBLEM 3 %%%%%%%%%%%%%%%%%%%%%%%%%%%%%%%%%%
\section*{Problem 3}

Consider a 1000 MWE reactor with a 33\% efficiency conversion from MWT to MWE. 
What is the minimum volume of UO$_2$, enriched to 3 (atom) \% $^{235}$U that could theoretically supply the yearly energy production of this reactor. 
Treat energy contributions as coming only from the fission of $^{235}$U. 
These fission events release about 200 MeV with 95\% of that energy staying in the reactor.



\section*{Problem 3 Solution}


We need to find the minimum required volume of UO$_2$, $V$, so let's begin by expressing that volume of UO$_2$ in terms of the number of UO$_2$ molecules it contains. 

First, we use $M_{\text{UO}_2}$ as the total mass of UO$_2$ and $\rho_{\text{UO}_2}$ as the density of UO$_2$ (about 10.4 g/cm$^3$ for our purposes)
$$ V = \frac{M_{\text{UO}_2}}{\rho_{\text{UO}_2}} $$
Then, we can write the mass of UO$_2$ as the number of UO$_2$ molecules, $N_{\text{UO}_2}$, times the mass of a UO$_2$ molecule, $m_{\text{UO}_2}$. Since we know that the UO$_2$ is enriched such that 3\% of the molecules contain $^{235}$U, we can say $m_{\text{UO}_2} = 0.03m_{\text{U5}} + 0.97m_{\text{U8}} + 2m_{\text{O}}$. Then
$$ M_{\text{UO}_2} = N_{\text{UO}_2}(0.03m_{\text{U5}} + 0.97m_{\text{U8}} + 2m_{\text{O}}) $$
We can use this equality in our total volume equation to find
\begin{equation}
\label{vol}
V = \frac{N_{\text{UO}_2}(0.03m_{\text{U5}} + 0.97m_{\text{U8}} + 2 m_{\text{O}})}{\rho_{\text{UO}_2}}
\end{equation}
Now we need to relate this volume, in terms of number of UO$_2$ molecules, to the electric power produced by the reactor. Say $P_E$ is the electric power, $P_T$ is thermal power, and $\varepsilon$ is the efficiency. 
$$ P_E = \varepsilon P_T $$
The total energy produced by the reactor in time $t$ can be found by simply multiplying the power by the time of production.
$$ E = P_T t $$
or equivalently
$$ P_T = \frac{E}{t} .$$
The energy from one fission is $E_f$ and the energy harnessed from one fission is $0.95E_f$. The total energy harnessed can then be expressed in terms of the number of fissions, $N_f$ and the energy captured per event,
$$ E = 0.95E_f N_f $$
If we consider the extreme and highly unrealistic case that \textit{all} $^{235}$U atoms fission, then the number of $^{235}$U atoms required is $N_{\text{U5}} = N_f$. Since $^{235}$U is 3\% of the total uranium by atom, then $N_{\text{U5}} = 0.03N_{\text{U}}$. Taking this one step further, there is a one-to-one ratio of uranium atoms to UO$_2$ molecules, so $N_{\text{U}} = N_{\text{UO}_2}$.

We combine these facts together to relate the electric power produced to the number of uranium dioxide molecules required.
\begin{align*} 
P_E	&= \varepsilon P_T \\
	&= \varepsilon \frac{E}{t} \\
	&= \frac{0.95\varepsilon E_f N_f}{t} \\
	&= \frac{(0.95)(0.03)\varepsilon E_f N_{\text{UO}_2}}{t} \\
\end{align*}
We solve for $N_{\text{UO}_2}$ to get
\begin{equation}
\label{num}
N_{\text{UO}_2} = \frac{P_E t}{(0.95)(0.03)\varepsilon E_f}
\end{equation}

Now, we use equation (\ref{num}) for the number of UO$_2$ molecules required for the given power in equation (\ref{vol}) for the volume of UO$_2$ required. We find
$$ V = \frac{P_E t(0.03m_{\text{U5}} + 0.97m_{\text{U8}} + 2 m_{\text{O}})}{(0.95)(0.03)\varepsilon E_f\rho_{\text{UO}_2}} $$
Finally, we can include our values. We note the following
\begin{itemize}
\item $P_E = 1000\text{ MWE} = 1000\text{ J/s}$
\item $t = 1\text{ yr} = 31,557,600\text{ s}$
\item $E_f = 200\text{ MeV} = 3.204 \times 10^{-11}$ J
\item $m_{\text{U5}} = 235.044\text{ amu} = 3.9028 \times 10^{-22}$ g
\item $m_{\text{U8}} = 238.051\text{ amu} = 3.9528 \times 10^{-22}$ g
\item $m_{\text{O}} = 15.995\text{ amu} = 2.656 \times 10^{-23}$ g
\end{itemize}
$$ V = \frac{(1000\text{ J/s})(31557600\text{ s})(0.03(3.9028 \times 10^{-22}\text{ g}) + 0.97(3.9528 \times 10^{-22}\text{ g}) + 2(2.656 \times 10^{-23}\text{ g}))}{(0.95)(0.03)(0.33)(3.204 \times 10^{-11}\text{ J})(10.41\text{ g/cm}^3)} $$
$$\boxed{ V = 4.51\text{ cm}^3 }$$



\newpage
%%%%% PROBLEM 4 %%%%%
%%%%%%%%%%%%%%%%%%%%%%%%%%%%%%%%%% PROBLEM 4 %%%%%%%%%%%%%%%%%%%%%%%%%%%%%%%%%%
\section*{Problem 4}

Given UO$_2$ with $2.5 \times 10^{21}$ atoms/cm$^3$ of $^{235}$U and $2.0\times10^{22}$ atoms/cm$^3$ of $^{238}$U, find the partial densities of $^{235}$U, $^{238}$U, and O, and determine the enrichment.



\section*{Problem 4 Solution}


The number density of material $i$ can be found using the formula
$$ n_i = \frac{\rho_i N_A}{M_i} $$
where $\rho_i$ is the partial density of material $i$, $M_i$ is the molar mass of $i$, and $N_A$ is Avogadro's number. We can reverse this formula to find the partial density of a material from the atomic density.
$$ \rho_i = \frac{N_i M_i}{N_A} $$
We can use this formula for the three isotopes we are considering. We also note the following:
\begin{itemize}
\item $n_{\text{O}} = 2(n_{\text{U5}} + n_{\text{U5}})$
\item $M_{\text{U5}} = 235.04\text{ g/mol}$
\item $M_{\text{U8}} = 238.05\text{ g/mol}$
\item $M_{\text{O}} = 16.00\text{ g/mol}$
\end{itemize}

\begin{multicols}{2}
$$ \rho_{\text{U5}} = \frac{n_{\text{U5}}  M_{\text{U5}} }{N_A} $$
$$ \rho_{\text{U5}} = \frac{(2.5\times10^{21}\text{ atoms/cm}^3)(235.04\text{ g/mol}) }{6.022\times10^{23}\text{ atoms/mol}} $$
$$\boxed{ \rho_{\text{U5}} = 0.976\text{ g/cm}^3 }$$


$$ \rho_{\text{U8}} = \frac{n_{\text{U8}}  M_{\text{U8}} }{N_A} $$
$$ \rho_{\text{U8}} = \frac{(2.0\times10^{22}\text{ atoms/cm}^3)(238.05\text{ g/mol}) }{6.022\times10^{23}\text{ atoms/mol}} $$
$$\boxed{ \rho_{\text{U8}} = 7.91\text{ g/cm}^3 }$$

\end{multicols}

$$ \rho_{\text{O}} = \frac{n_{\text{O}}  M_{\text{O}} }{N_A} $$
$$ \rho_{\text{O}} = \frac{2(n_{\text{U5}} + n_{\text{U5}})  M_{\text{O}} }{N_A} $$
$$ \rho_{\text{O}} = \frac{2(2.25\times10^{22}\text{ atoms/cm}^{3})(16.00\text{ g/mol}) }{6.022\times10^{23}\text{ atoms/mol}} $$
$$\boxed{ \rho_{\text{O}} = 1.20\text{ g/cm}^3 }$$

The enrichment of $^{235}$U is therefore
$\frac{\rho_{\text{U5}}}{\rho_{\text{U5}}+\rho_{\text{U8}}} = 0.1098 \quad\Rightarrow \quad\boxed{10.98\%}$



\end{document}

